% Définition du Process Hitting + sortes coopératives

\begin{frame}[c]
  \frametitle{Standard Process Hitting}
  \framesubtitle{\tcite{\cpmrtcsb}}

\tval{Standard Process Hitting} is:
\begin{itemize}
  \item Well-adapted to the modeling of BRNs
  \item An \tval{atomistic and qualitative} modeling (explicit \& discrete expression levels)
  \item \tval{Simple but powerful} dynamics (constraints on the form of actions)
\end{itemize}

\pause
\bigskip
Previously developed tools:
\begin{itemize}
  \item \tval{Reachability analysis} by abstract interpretation
  \item Fixed points enumeration
  \item Stochastic parameters
\end{itemize}

\medskip
\f Well-adapted formalism to study \tval{large BRNs}

\pause
\bigskip
Several missing features:
\begin{itemize}
  \item Faulty representation \tval{cooperations}
  \item \tval{Possible enrichment} of the expressivity\\
    \quad \f Which requires to adapt the previous tools
\end{itemize}

\end{frame}



\begin{frame}[t]
  \frametitle{Standard Process Hitting}
  \framesubtitle{\tcite{\cpmrtcsb}}

% 1 : Sortes
\only<1>{
\tikzstyle{process}=[circle,minimum size=15pt,font=\footnotesize,inner sep=1pt]
\tikzstyle{tick label}=[color=white, font=\footnotesize]
\tikzstyle{tick}=[transparent]
\tikzstyle{hit}=[transparent]
\tikzstyle{selfhit}=[transparent, min distance=30pt,curve to]
\tikzstyle{bounce}=[transparent]
\tikzstyle{hlhit}=[transparent]
\begin{center}\scalebox{\scaleex}{
\begin{tikzpicture}
  \exphdef
\end{tikzpicture}
}\end{center}
}

% 2 : Processus
\only<2>{
\tikzstyle{process}=[circle,draw,minimum size=15pt,font=\footnotesize,inner sep=1pt]
\tikzstyle{tick label}=[font=\footnotesize]
\tikzstyle{tick}=[densely dotted]
\tikzstyle{hit}=[transparent]
\tikzstyle{selfhit}=[transparent, min distance=30pt,curve to]
\tikzstyle{bounce}=[transparent]
\tikzstyle{hlhit}=[transparent]
\begin{center}\scalebox{\scaleex}{
\begin{tikzpicture}
  \exphdef
\end{tikzpicture}
}\end{center}
}

% 3 : États
\only<3>{
\tikzstyle{hit}=[transparent]
\tikzstyle{selfhit}=[transparent, min distance=30pt,curve to]
\tikzstyle{bounce}=[transparent]
\tikzstyle{hlhit}=[transparent]
\begin{center}\scalebox{\scaleex}{
\begin{tikzpicture}
  \exphdef

  \TState{3}{a_0,b_1,z_0}
\end{tikzpicture}
}\end{center}
}

% 4 : Actions
\only<4->{
\tikzstyle{tick}=[densely dotted]
\tikzstyle{hit}=[->,>=angle 45]
\tikzstyle{selfhit}=[min distance=30pt,curve to]
\tikzstyle{bounce}=[densely dotted,>=stealth',->]
\tikzstyle{hlhit}=[very thick]
\begin{center}\scalebox{\scaleex}{
\begin{tikzpicture}
\exphdef
  \TState{4-5}{a_0,b_1,z_0}
  \TState{6}{a_0,b_1,z_1}
  \TState{7}{a_1,b_1,z_1}
  \TState{8}{a_1,b_1,z_2}

  \only<5>{
    \THit{b_1}{hl}{z_0}{.west}{z_1}
    \path[bounce,bend left,hl] \TBounce{z_0}{}{z_1}{.south};
  }
  \only<6>{
    \THit{a_0}{out=250,in=200,selfhit,hl}{a_0}{.west}{a_1}
    \path[bounce,bend left,hl] \TBounce{a_0}{}{a_1}{.south};
  }
  \only<7>{
    \THit{a_1}{hl}{z_1}{.west}{z_2}
    \path[bounce,bend left,hl] \TBounce{z_1}{}{z_2}{.south};
  }
\end{tikzpicture}
}\end{center}
}

\medskip
\begin{liste}
  \item \tval{Sorts}: components \qex{$a$, $b$, $z$}
\pause[2]
  \item \tval{Processes}: local states / discrete expression levels \qex{$z_0$, $z_1$, $z_2$}
\pause[3]
  \item \tval{States}: sets of active processes%
  \only<3-5>{\qex{$\PHetat{a_0, b_1, z_0}$}}%
  \only<6>{\qex{$\PHetat{a_0, b_1, z_1}$}}%
  \only<7>{\qex{$\PHetat{a_1, b_1, z_1}$}}%
  \only<8>{\qex{$\PHetat{a_1, b_1, z_2}$}}%
\pause[4]
  \item \tval{Actions}: dynamics \qex{\only<5>{\underline}{$\PHfrappe{b_1}{z_0}{z_1}$}, \only<6>{\underline}{$\PHfrappe{a_0}{a_0}{a_1}$}, \only<7>{\underline}{$\PHfrappe{a_1}{z_1}{z_2}$}}
\end{liste}
\end{frame}



\begin{frame}
  \frametitle{Cooperations}
  \framesubtitle{\tcite{\cpmrtcsb}}

\begin{center}\scalebox{\scaleex}{
\begin{tikzpicture}
  \exphcoop
  
  \TState{9}{a_1,b_1}
  \TState{10}{a_1,b_1,ab_0}
  \TState{11}{a_1,b_1,ab_1}
  \TState{12}{a_1,b_1,ab_3}
  
  \node at (a_1.center) {\textasteriskcentered};
  \node at (b_1.center) {\textasteriskcentered};
  \node<5-> at (ab_3.center) {\textasteriskcentered};
  
  \only<9>{
    \node[hl process] at (ab_0.center) {};
    \node[hl process] at (ab_1.center) {};
    \node[hl process] at (ab_2.center) {};
    \node[current process] at (ab_3.center) {};
  }
  
  \only<10>{
    \THit{b_1}{hl}{ab_0}{.210}{ab_1}
    \path[bounce,bend left,hl] \TBounce{ab_0}{}{ab_1}{.240} ;
  }
  
  \only<11>{
    \THit{a_1}{hl}{ab_1}{.160}{ab_3}
    \path[bounce,bend left,hl] \TBounce{ab_1}{}{ab_3}{.south west} ;
  }
\end{tikzpicture}
}\end{center}

\medskip
\begin{liste}
  \item \tval{Cooperation} between \ex{$a_1$} and \ex{$b_1$}: \qex{$\PHfrappe{\underline{a_1 \wedge b_1}}{z_0}{z_1}$}
\pause[5]
  \item Solution: a \tval{cooperative sort} \qex{$ab$} \quad to express \qex{$\underline{a_1 \wedge b_1}$}
\pause[9]
  \item Each configuration is represented by one process \qex{$\underline{a_1 \wedge b_1} \Rightarrow ab_{11}$}
%\pause[15]
%  \item Advantage: regular sort; drawbacks: complexity, temporal shift
\end{liste}
\end{frame}

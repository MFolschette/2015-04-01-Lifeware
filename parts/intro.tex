% Intro

\begin{frame}[c]
  \frametitle{The Modeling/Analysis duality}

Modeling a system is the first step towards its comprehension

\begin{center}
\begin{tikzpicture}
  \node[ellipse, fill=blue!20] (m) at (-1.5, 0) {Modeling};
  \node[ellipse, fill=violet!20] (a) at (1.5, 0) {Analysis};
  \uncover<2->{ \path[->, shorten >=1em, shorten <=1em] (a) edge[ultra thick, bend left] (m); }
  \uncover<3->{ \path[->, shorten >=1em, shorten <=1em] (m) edge[ultra thick, bend left] (a); }
\end{tikzpicture}
\end{center}

\pause[2]
The required analysis has an impact on modeling
\begin{itemize}
  \item The modeling tools must be adapted to the observed properties
\end{itemize}

\pause[3]
\medskip
Modeling choices have an impact on the results of the analysis
\begin{itemize}
  \item The level of details changes the quantity of obtained info
  \item The size of the model increases the analysis duration
\end{itemize}

\pause[4]
\medskip
\begin{center}
  \tval{The modeling and analysis steps of a system are strongly linked}
\end{center}

\end{frame}



\begin{frame}[c]
\frametitle{Overview of This Presentation}

\tval{State of the Art} of the modeling of biological regulatory networks
\begin{itemize}
  \item Discrete asynchronous representations and Thomas modeling
  \item Standard Process Hitting
\end{itemize}

\pause
\bigskip
\tval{Enriching} the Process Hitting
\begin{itemize}
  \item Integration of temporal constraints
  \item Synchronicity between actions
  \item[] \quad \f Adding of priorities, neutralizing edges or synchronous actions
\end{itemize}

\pause
\bigskip
\tval{Analysis} of the Process Hitting
\begin{itemize}
  \item Correction of the cooperative sorts
  \item Static analysis of reachability
  \item Equivalences and links with other formalisms
\end{itemize}

\end{frame}

\documentclass[fleqn,8pt,t]{beamer}

%\usepackage[french]{babel}
\usepackage[utf8]{inputenc}
\usepackage[T1]{fontenc}
%\usepackage{french} % Sommaire en début de document
%\usepackage[top=2cm, bottom=2cm, left=2cm, right=2cm]{geometry} % Marges

\usepackage{amsmath} % Maths
\usepackage{amsfonts} % Maths
\usepackage{amssymb} % Maths
\usepackage{stmaryrd} % Maths (crochets doubles)

%\usepackage{listings} % Mise en forme du code (pour Hoare) ## À REVOIR ###
%\usepackage{ifthen} % Structures If Then Else
\usepackage{theorem} % Styles supplémentaires pour théorèmes
\usepackage{url}
\usepackage{xcolor,colortbl} % Cellules colorées
\usepackage{array}  % Tableaux évolués
\usepackage{multirow}  % Pour des colonnes sur plusieurs lignes

%\usepackage{enumerate} % Changer les puces des listes d'énumération
%\usepackage{setspace} % Changer les interlignes

%\usepackage{subfig} % Créer des sous-figures
%\usepackage{graphicx} % Importer des images
\usepackage{tikz}

\usepackage{ulem}  % Pour l'attribut barré

\usepackage{comment}

% Police
\usepackage{lmodern}
%\usepackage{libertine}
% \usepackage{ziffer}     % Typographie française pour les nombres
% \ZifferAus
%\ZifferAn


\input{macros/macros}
\input{macros/macros-ph}
\input{macros/macros-abstr}
\input{macros/tikzstyles}



% Commande À FAIRE
\usepackage{color} % Couleurs du texte
%\newcommand{\afaire}[1]{\textcolor{red}{[À FAIRE : #1]}}
\newcommand{\todo}[1]{\textcolor{red}{<[[#1]]>}}



% \colorlet{couleurtheme}{gray}  % Couleur principale du thème
% \colorlet{couleurcit}{gray}  % Couleur des citations
% \colorlet{couleurex}{blue}  % Couleur des citations
% \colorlet{couleurliens}{darkblue}  % Couleur des citations

\definecolor{couleurtheme}{rgb}{0.4,0.4,.8}
\definecolor{couleuremph}{rgb}{0.3,0.3,.9}
%\definecolor{couleurtheme}{rgb}{0,0.7,0.8}
%\colorlet{couleurtheme}{darkcyan}  % Couleur principale du thème
\colorlet{couleurcit}{gray}  % Couleur des citations
\colorlet{couleurmescit}{violet}  % Couleur des citations
\colorlet{couleurex}{blue}  % Couleur des citations
\colorlet{couleurliens}{darkblue}  % Couleur des citations
%\definecolor{couleurtitre}{rgb}{0.54,0.8,0.9}
\colorlet{couleurtitre}{couleurtheme!50}  % Cadre du titre

\usetheme{Pittsburgh}   % Thème général
\usefonttheme{default}  % Thème de polices
\setbeamertemplate{navigation symbols}{}  % Pas de menu de navigation
%\setbeamertemplate{itemize item}[x]   % Puces des listes

\usecolortheme[named=couleurtheme]{structure}    % Couleur de la structure : titres et puces
%\setbeamercolor{normal text}{bg=black,fg=white}  % Couleur du texte
\setbeamercolor{item}{fg=couleurtheme}           % Couleur des puces
%\setbeamercolor{item projected}{fg=black}        % Couleur des recouvrements
%\setbeamercolor{alerted text}{fg=yellow}         % ?

\setbeamerfont{frametitle}{size=\Large}  % Police des titres


% Flèche grise
\newcommand{\f}{\textcolor{couleurtheme}{\textbf{$\rightarrow$\ }}}
\newcommand{\cth}[1]{\textcolor{couleurtheme}{#1}}

% Environnement liste avec flèches
\newenvironment{fleches}{%
\begin{list}{}{%
\setlength{\labelwidth}{1em}% largeur de la boîte englobant le label
\setlength{\labelsep}{0pt}% espace entre paragraphe et l’étiquette
%\setlength{\itemsep}{1pt}
%\setlength{\leftmargin}{\labelwidth+\labelsep}% marge de gauche
\renewcommand{\makelabel}{\f}%
}}{\end{list}}

% Liste sans puce
\newenvironment{liste}{%
\begin{list}{}{%
\setlength{\labelwidth}{0em}% largeur de la boîte englobant le label
\setlength{\labelsep}{0pt}% espace entre paragraphe et l’étiquette
\setlength{\leftmargin}{0em}% marge de gauche
%\renewcommand{\makelabel}{\f}%
}}{\end{list}}

% Style des exemples
\newcommand{\ex}[1]{\textcolor{couleurex}{#1}}
\newcommand{\qex}[1]{\quad \ex{#1}}
\newcommand{\rex}[1]{\hfill \ex{#1}}
\newcommand{\redex}[1]{\textcolor{red}{#1}}

\newcommand{\lien}[1]{\textcolor{couleurliens}{\underline{\url{#1}}}}

\newcommand{\console}[1]{\textcolor{darkgray}{#1}}

\newcommand{\emphcolor}[1]{\textcolor{couleuremph}{#1}}

% Style des citations
\newcommand{\tscite}[1]{\textcolor{couleurcit}{#1}}
\newcommand{\tcite}[1]{\textcolor{couleurcit}{[#1]}}
\newcommand{\tcitebullet}{~~$\textcolor{couleurtheme}{\bullet}$~}

\newcommand{\vol}{Vol.\ }
%\newcommand{\no}{No.\ }
%\newcommand{\cad}{c.-à-d.\ }
\newcommand{\cad}{\ie}
\newcommand{\ie}{i.e.\ }



% Style de texte mis en valeur
\newcommand{\tval}[1]{\textbf{#1}}

% Un vrai symbole pour l'ensemble vide
\renewcommand{\emptyset}{\varnothing}

% Pour définir la conférence et son nom court
\newcommand{\conference}[2]{\def\theconference{#2}
\def\insertshortconference{\ifthenelse{\equal{#1}{-}}{#2}{\ifthenelse{\equal{#1}{}}{#2}{#1}}}}



\newcommand{\thedate}{2014/10/08}
\date{\thedate}
\conference{Lifeware seminar}{Public seminar of team Lifeware}
\title[Modeling and analysis of large RN with the PH framework]%
  {Modeling and analysis of large regulatory networks with the Process Hitting framework}
\author{Maxime FOLSCHETTE}




\setbeamertemplate{footline}{\color{gray}%
\scriptsize
\quad\strut%
\insertauthor%
\hfill%
\insertframenumber/\inserttotalframenumber%
\hfill%
\insertshortconference{} --- \thedate\quad\strut
}


\newcommand{\headersep}{$\circ$} % \bullet \triangleright

\setbeamertemplate{headline}{\color{gray}%
\vskip0.3em%
\quad\strut%
{\scriptsize\color{black}%
% Gris si une section existe
\ifthenelse{\equal{\thesection}{0}}{}{%
\ifthenelse{\equal{\lastsection}{x}}{}{%
\color{gray}%
}}%
\insertshorttitle
\ifthenelse{\equal{\thesection}{0}}{}{%
\ifthenelse{\equal{\lastsection}{x}}{}{%
~\headersep{} %
% Gris si une sous-section existe
\ifthenelse{\equal{\thesubsection}{0}}{\color{black}}{%
\ifthenelse{\equal{\lastsubsection}{x}}{\color{black}}{%
\color{gray}%
}}%
\insertsectionhead%
%
\ifthenelse{\equal{\thesubsection}{0}}{}{%
\ifthenelse{\equal{\lastsubsection}{x}}{}{%
~\headersep{} \color{black}\insertsubsectionhead%
%
}}}}}%
\vskip-5ex%
}



\def \scaleex {0.85}
\def \scaleminiex {0.6}
\def \scaleinf {0.6}

\colorlet{colorb}{blue}
\colorlet{colora1}{yellow}
\colorlet{colora0}{green}
\colorlet{colora1font}{darkyellow}
\colorlet{colora0font}{darkgreen}

\colorlet{exanswer}{blue}
\colorlet{colorgray}{lightgray}



\begin{document}

\begin{frame}[plain,label=title]

% Cadre de titre
\begin{center}
%\vspace{.7cm}
\vfill
\textcolor{couleurtheme}{\fbox{\noindent%
\setbeamercolor{postit}{fg=black,bg=couleurtitre}%
\begin{beamercolorbox}[sep=1.5em]{postit}
\centering
\Large
\textbf{%
{\normalsize\theconference{}}\\~\\%
\inserttitle
}
\end{beamercolorbox}}}

% Auteurs et instituts
\par
\medskip
\bigskip
\normalsize
\tval{Maxime FOLSCHETTE}

\medskip
\footnotesize
MeForBio / IRCCyN / École centrale de Nantes (Nantes, France)

\texttt{maxime.folschette@irccyn.ec-nantes.fr}

\url{http://maxime.folschette.name/}

\bigskip

\normalsize
\thedate
\end{center}

\scriptsize
\bigskip
%Jury de thèse :
%\\

%\president{Mme}{Prénom}{Nom}{Titre}{Établissement}
%\guest{Mme}{Prénom}{Nom}{Titre}{Établissement}

\begin{tabular}{r@{\ \ }l}
\textbf{Rapporteurs :}
& Jean-Paul COMET, Professeur des universités,
    Université de Nice -- Sophia Antipolis \\
& Anne SIEGEL, Directrice de recherche CNRS,
    IRISA (CNRS \& Université Rennes 1), Inria Rennes \vspace*{1em} \\
\textbf{Examinateurs :}
& Mireille RÉGNIER, Directrice de recherche Inria,
    École polytechnique \& Université Paris-Sud 11 \\
& Denis THIEFFRY, Professeur des universités,
    École normale supérieure \vspace*{1em} \\
\textbf{Directeur de thèse :}
& Olivier ROUX, Professeur des universités,
    École centrale de Nantes \\
\textbf{Co-encadrant de thèse :}
& Morgan MAGNIN, Maître de conférences,
    École centrale de Nantes
\end{tabular}

\vfill

\end{frame}



\input{parts/ex.tex}

% Références des modèles
\newcommand{\cmodels}{\bigskip
\quad\tval{\ex{egfr20}} : Epithelial Growth Factor Receptor (20 components) \tcite{Sahin \textit{et al.}, 2009}\\
\quad\tval{\ex{egfr104}} : Epithelial Growth Factor Receptor (104 components) \tcite{Samaga \textit{et al.}, 2009}\\
\quad\tval{\ex{tcrsig40}} : T-Cell Receptor (40 composants) \tcite{Klamt \textit{et al.}, 2006}\\
\quad\tval{\ex{tcrsig94}} : T-Cell Receptor (94 composants) \tcite{Saez-Rodriguez \textit{et al.}, 2007}\\}
% \quad\tval{\ex{egfr20}} : Récepteur de croissant épidermique (20 composants) \tcite{Özgür Sahin \textit{et al.}, 2009}\\
% \quad\tval{\ex{egfr104}} : Récepteur de croissant épidermique (104 composants) \tcite{Regina Samaga \textit{et al.}, 2009}\\
% \quad\tval{\ex{tcrsig40}} : Récepteur de lymphocyte T (40 composants) \tcite{Steffen Klamt \textit{et al.}, 2006}\\
% \quad\tval{\ex{tcrsig94}} : Récepteur de lymphocyte T (94 composants) \tcite{Julio Saez-Rodriguez \textit{et al.}, 2007}\\}

% Under-approximation of Reachability in Multivalued Asynchronous Networks
\newcommand{\cfpmrcsbio}{Folschette \textit{et al.} in \textit{Workshop on Interactions between Computer Science and Biology}, 2013}
% Refining dynamics of gene regulatory networks in a stochastic $\pi$-calculus framework
\newcommand{\cpmrtcsb}{Paulevé \textit{et al.} in \textit{Transactions on Computational Systems Biology}, 2011}
% Static analysis of biological regulatory networks dynamics using abstract interpretation
\newcommand{\cpmrmscs}{Paulevé \textit{et al.} in \textit{Mathematical Structures in Computer Science}, 2012}
% Concretizing the Process Hitting into Biological Regulatory Networks
\newcommand{\cfpimrcmsb}{Folschette \textit{et al.} in \textit{Computational Methods in Systems Biology}, 2012}
% Formal Methods for Modeling Biological Regulatory Networks
%\newcommand{\crcbmfma}{Richard \textit{et al.} in Modern Formal Methods and App., 2006}
% Semantics of Biological Regulatory Networks
%\newcommand{\bccdmr}{Bernot \textit{et al.} in Concurrent Models in Molecular Biology, 2007}
% Negative circuits and sustained oscillations in asynchronous automata networks
%\newcommand{\rcite}{Richard in Advances in Applied Mathematics, 2010}
% Thèse de Loïc
\newcommand{\paulevephd}{Paulevé (PhD thesis), 2011}
% R. Thomas' logical method
\newcommand{\citerichard}{Richard, Comet, Bernot (tutorial), 2008}

\newcommand{\citerichardcomet}{Richard, Comet in \textit{Discrete Applied Mathematics}, 2007}
\newcommand{\citeremy}{Remy, Ruet, Thieffry in \textit{Advances in Applied Mathematics}, 2008}
\newcommand{\citesmbionet}{Bernot, Comet, Richard, Guespin in \textit{Journal of Theoretical Biology}, 2004}
\newcommand{\citeito}{Ito, Izumi, Hagihara, Yonezaki in \textit{BioInformatics and BioEngineering}, 2010}
\newcommand{\citethomas}{Thomas in \textit{Journal of Theoretical Biology}, 1973}
\newcommand{\citekauffman}{Kauffman in \textit{Journal of Theoretical Biology}, 1969}



\newcommand{\planPHstandard}[1][]{%
  \draw[very thick, draw=gray, fill=gray!10, #1] (0,-2.5) ellipse (1.7 and 1)
    node[text width=4cm, align=center] {Standard\\Process Hitting};}
\newcommand{\planPHp}[1][]{%
  \draw[very thick, draw=blue, fill=blue!10, #1] (0,2) ellipse (2.5 and 1.5)
    node[text width=4cm, align=center, yshift=.5cm] {Process Hitting\\with classes of priority};}
\newcommand{\planPHcanonique}[1][]{%
  \draw[very thick, draw=red, fill=red!10, #1] (0,1.3) ellipse (1.8 and .5)
    node[text width=4cm, align=center, yshift=0cm] {Canonical\\Process Hitting};}
\newcommand{\planPHan}[1][]{%
  \draw[very thick, draw=darkgreen, fill=darkgreen!10, #1] (-3.5,0) ellipse (1.7 and 1)
    node[text width=4cm, align=center] {Process Hitting\\with neutralizing edges};}
\newcommand{\planPHmult}[1][]{%
  \draw[very thick, draw=violet, fill=violet!10, #1] (3.5,0) ellipse (1.7 and 1)
    node[text width=4cm, align=center] {Process Hitting\\with synchronous actions};}
\newcommand{\planPHstandardligne}{
  \draw[very thick, draw=gray, bend left=30] (-5,-2.7) edge (5,-2.7);}



\section{Introduction}
% Intro

\begin{frame}[c]
  \frametitle{The Modeling/Analysis duality}

Modeling a system is the first step towards its comprehension

\begin{center}
\begin{tikzpicture}
  \node[ellipse, fill=blue!20] (m) at (-1.5, 0) {Modeling};
  \node[ellipse, fill=violet!20] (a) at (1.5, 0) {Analysis};
  \uncover<2->{ \path[->, shorten >=1em, shorten <=1em] (a) edge[ultra thick, bend left] (m); }
  \uncover<3->{ \path[->, shorten >=1em, shorten <=1em] (m) edge[ultra thick, bend left] (a); }
\end{tikzpicture}
\end{center}

\pause[2]
The required analysis has an impact on modeling
\begin{itemize}
  \item The modeling tools must be adapted to the observed properties
\end{itemize}

\pause[3]
\medskip
Modeling choices have an impact on the results of the analysis
\begin{itemize}
  \item The level of details changes the quantity of obtained info
  \item The size of the model increases the analysis duration
\end{itemize}

\pause[4]
\medskip
\begin{center}
  \tval{The modeling and analysis steps of a system are strongly linked}
\end{center}

\end{frame}



\begin{frame}[c]
\frametitle{Overview of This Presentation}

\tval{State of the Art} of the modeling of biological regulatory networks
\begin{itemize}
  \item Discrete asynchronous representations and Thomas modeling
  \item Standard Process Hitting
\end{itemize}

\pause
\bigskip
\tval{Enriching} the Process Hitting
\begin{itemize}
  \item Integration of temporal constraints
  \item Synchronicity between actions
  \item[] \quad \f Adding of priorities, neutralizing edges or synchronous actions
\end{itemize}

\pause
\bigskip
\tval{Analysis} of the Process Hitting
\begin{itemize}
  \item Correction of the cooperative sorts
  \item Static analysis of reachability
  \item Equivalences and links with other formalisms
\end{itemize}

\end{frame}


\section{State of the Art of Modeling}
% Abstractions

\begin{frame}[c]
  \frametitle{Abstractions of the Representation}

\begin{center}
  \includegraphics[width=.5\textwidth]{figs/protein.png}
  \uncover<2>{
    \begin{tikzpicture}
      \path[use as bounding box] (-3, 1) rectangle (0, -2);
      \node at (-1.5,.7) (ga) {\underline{Gene a}};
      \node at (-1.5,-0.7) (pa) {\underline{Protein a}};
      \node[draw=none] at (-2.5,0) {$\Rightarrow$};
      \path[draw,->] (ga) -- (pa);
      \node[draw=none] at (-0.5,0) {$\Rightarrow$};
    \end{tikzpicture}}
  \uncover<2>{
    \scalebox{2}{
    \begin{tikzpicture}[adn]
      \path[use as bounding box] (-.3, .5) rectangle (0.5, -1);
      \node (a) {a};
    \end{tikzpicture}}
  }
\end{center}

\end{frame}



\begin{frame}[c]
  \frametitle{Discretization and Asynchronism}
  \framesubtitle{\tcite{\citerichard}}

\begin{center}
  \includegraphics[width=.75\textwidth]{figs/seuils1.png}
\end{center}

\begin{tikzpicture}[adn]
  \path[use as bounding box] (-1.5,-5) rectangle (2.7,-5);
  
  \node[inner sep=0] (a) at (0,0) {a};
  \node[inner sep=0] (b) at (2,0) {b};
  
  \path node[elabel, below=-1em of b] {$\segm{0}{1}$};
  \path<2-> node[elabel, below=-1em of a] {$\segm{0}{2}$};
  
  \path<1> (b) edge[very thick,bend right=15] (a);
  \path<2-> (b) edge[bend right=15] (a);
  
  \path<2-> (a) edge[very thick,bend right=15] (b)
    (a) edge[very thick,loop left] (a);
  
  \path<1>[fill=white] (3.3,-4.2) rectangle (8,.8);
\end{tikzpicture}

\vspace*{-2.5em}
\pause[3]
\begin{itemize}
  \item Unknown real values of concentrations or continuous activity levels\\
    \quad \f Abstracted as thresholds or \tval{discrete levels}
\pause
  \item Continuous variations of the real values\\
    \quad \f \tval{Unitary} dynamics
\pause
  \item Simultaneous crossings of two thresholds never occurs\\
    \quad \f \tval{Asynchronous} dynamics
\end{itemize}

\end{frame}

% Définition du Réseaux Discrets Asynchrones

\newcommand{\Fadn}{\mathbb{F}}
\newcommand{\Eadn}{\mathbb{E}}
\newcommand{\SGadn}{\mathrm{G}}

\begin{frame}[t]
  \frametitle{Discrete Networks / Thomas Modeling}
  \framesubtitle{\tcite{\citekauffman}\\\tcite{\citethomas}}

\begin{itemize}
  \item A set of components \qex{$N = \{ a, b, z \}$}
\uncover<2->{
  \item A set of discrete expression levels for each component \qex{$z \in \Fadn^z = \segm{0}{2}$}
  \item The set of global states \qex{$\Fadn = \Fadn^a \times \Fadn^b \times \Fadn^z$}
}
\uncover<3->{
  \item An evolution function for each component \qex{$f^z : \Fadn \rightarrow \Fadn^z$}
}
\uncover<4->{
  \item Signs and thresholds on the edges \qex{$a \xrightarrow{+1}z$}
\end{itemize}
}

\uncover<3->{
\begin{center}
\begin{tabular}{ccc}
%  \ex{$f^a = \neg b$} & \ex{$f^b = b \vee \neg a$} & \ex{$f^z = a + b$} \vspace{.5em}\\
  \begin{tabular}[t]{c|c}
    $b$ & $f^a(b)$ \\
  \hline
    $0$ & $\mathbf{1}$ \\
    $1$ & $\mathbf{0}$ \\
  \end{tabular}
&
  \begin{tabular}[t]{cc|c}
    $a$ & $b$ & $f^b(a, b)$ \\
  \hline
    $0$ & $0$ & $\mathbf{1}$ \\
    $0$ & $1$ & $\mathbf{1}$ \\
    $1$ & $0$ & $\mathbf{0}$ \\
    $1$ & $1$ & $\mathbf{1}$
  \end{tabular}
&
  \begin{tabular}[t]{cc|c}
    $a$ & $b$ & $f^z(a, b)$ \\
  \hline
    $0$ & $0$ & $\mathbf{0}$ \\
    $0$ & $1$ & $\mathbf{1}$ \\
    $1$ & $0$ & $\mathbf{1}$ \\
    $1$ & $1$ & $\mathbf{2}$
  \end{tabular}
\end{tabular}
}

\bigskip

\begin{tikzpicture}[adn]
  \path[use as bounding box] (-0.7,-0.7) rectangle (2.5,2);
  \node[inner sep=0] (z) at (2,0.75) {z};
  \node[inner sep=0] (a) at (0,1.5) {a};
  \node[inner sep=0] (b) at (0,0) {b};
  \path<2->
    node[alabel, above=-1em of a] {$\segm{0}{1}$}
    node[alabel, below=-1em of b] {$\segm{0}{1}$}
    node[alabel, below=-1em of z] {$\segm{0}{2}$};
  \path<3->
    (a) edge[bend right=15] (b)
    (b) edge[bend right=15] (a)
    (b) edge[loop left] (b)
    (a) edge (z)
    (b) edge (z);
  \path<4->
    (a) edge[draw=none,bend right=15] node[alabel,left=-1pt] {$-1$} (b)
    (b) edge[draw=none,bend right=15] node[alabel,right=-3pt] {$-1$} (a)
    (b) edge[draw=none,loop left] node[alabel,left=-2pt] {$+1$} (b)
    (a) edge[draw=none] node[alabel,above=-2pt] {$+1$} (z)
    (b) edge[draw=none] node[alabel,below=-2pt] {$+1$} (z);
\end{tikzpicture}
\end{center}

\end{frame}



\begin{frame}[c]
  \frametitle{Analysis of Thomas Modeling}

The State graph is computed in a unitary and asynchronous fashion
\begin{center}
\scalebox{.8}{
\begin{tikzpicture}[stategraph]
  \node (000) {$\RRBetat{a_0,b_0,z_0}$};
  \node[right of=000] (001) {$\RRBetat{a_0,b_0,z_1}$};
  \node[right of=001] (002) {$\RRBetat{a_0,b_0,z_2}$};
  
  \node[below of=000, node distance=1.7cm] (100) {$\RRBetat{a_1,b_0,z_0}$};
  \node[right of=100] (101) {$\RRBetat{a_1,b_0,z_1}$};
  \node[right of=101] (102) {$\RRBetat{a_1,b_0,z_2}$};
  
  \node[at=(000), yshift=-.8cm, xshift=1.6cm] (010) {$\RRBetat{a_0,b_1,z_0}$};
  \node[right of=010] (011) {$\RRBetat{a_0,b_1,z_1}$};
  \node[right of=011] (012) {$\RRBetat{a_0,b_1,z_2}$};
  
  \node[at=(010), yshift=-1.7cm] (110) {$\RRBetat{a_1,b_1,z_0}$};
  \node[right of=110] (111) {$\RRBetat{a_1,b_1,z_1}$};
  \node[right of=111] (112) {$\RRBetat{a_1,b_1,z_2}$};
  
  \path
    (000) edge (100) edge (010)
    (001) edge (000) edge (101) edge (011)
    (002) edge (102) edge (012) edge (001)
    (010) edge (011)
  % (011)
    (012) edge (011)
    (100) edge (101)
  % (101)
    (102) edge (101)
    (110) edge (010) edge (111)
    (111) edge (011) edge (112)
    (112) edge (012)
  ;
\end{tikzpicture}
}
\end{center}
\f \tval{Exponential} size in the number of components

\pause
\bigskip
Some works all to link the structure of the model and some dynamic properties:
\begin{itemize}
  \item \tval{Thomas' conjectures} (conditions for multi-stationarity or sustained oscillations)
  \begin{itemize}
    \item Boolean case: \tcite{\citeremy}
    \item Multivalued case: \tcite{\citerichardcomet}
  \end{itemize}
\end{itemize}

\pause
\medskip
But reachability properties require to compute the whole state graph:\\
Example: \ex{From the initial state $(a, b, z) = (0, 0, 0)$, is it possible to reach $z = 2$?}
\begin{itemize}
  \item \tval{Temporal logics}
  \begin{itemize}
    \item CTL: \tcite{\citesmbionet}
    \item LTL: \tcite{\citeito}
  \end{itemize}
\end{itemize}
\end{frame}


% Définition du Process Hitting + sortes coopératives

\begin{frame}[c]
  \frametitle{Standard Process Hitting}
  \framesubtitle{\tcite{\cpmrtcsb}}

\tval{Standard Process Hitting} is:
\begin{itemize}
  \item Well-adapted to the modeling of BRNs
  \item An \tval{atomistic and qualitative} modeling (explicit \& discrete expression levels)
  \item \tval{Simple but powerful} dynamics (constraints on the form of actions)
\end{itemize}

\pause
\bigskip
Previously developed tools:
\begin{itemize}
  \item \tval{Reachability analysis} by abstract interpretation
  \item Fixed points enumeration
  \item Stochastic parameters
\end{itemize}

\medskip
\f Well-adapted formalism to study \tval{large BRNs}

\pause
\bigskip
Several missing features:
\begin{itemize}
  \item Faulty representation \tval{cooperations}
  \item \tval{Possible enrichment} of the expressivity\\
    \quad \f Which requires to adapt the previous tools
\end{itemize}

\end{frame}



\begin{frame}[t]
  \frametitle{Standard Process Hitting}
  \framesubtitle{\tcite{\cpmrtcsb}}

% 1 : Sortes
\only<1>{
\tikzstyle{process}=[circle,minimum size=15pt,font=\footnotesize,inner sep=1pt]
\tikzstyle{tick label}=[color=white, font=\footnotesize]
\tikzstyle{tick}=[transparent]
\tikzstyle{hit}=[transparent]
\tikzstyle{selfhit}=[transparent, min distance=30pt,curve to]
\tikzstyle{bounce}=[transparent]
\tikzstyle{hlhit}=[transparent]
\begin{center}\scalebox{\scaleex}{
\begin{tikzpicture}
  \exphdef
\end{tikzpicture}
}\end{center}
}

% 2 : Processus
\only<2>{
\tikzstyle{process}=[circle,draw,minimum size=15pt,font=\footnotesize,inner sep=1pt]
\tikzstyle{tick label}=[font=\footnotesize]
\tikzstyle{tick}=[densely dotted]
\tikzstyle{hit}=[transparent]
\tikzstyle{selfhit}=[transparent, min distance=30pt,curve to]
\tikzstyle{bounce}=[transparent]
\tikzstyle{hlhit}=[transparent]
\begin{center}\scalebox{\scaleex}{
\begin{tikzpicture}
  \exphdef
\end{tikzpicture}
}\end{center}
}

% 3 : États
\only<3>{
\tikzstyle{hit}=[transparent]
\tikzstyle{selfhit}=[transparent, min distance=30pt,curve to]
\tikzstyle{bounce}=[transparent]
\tikzstyle{hlhit}=[transparent]
\begin{center}\scalebox{\scaleex}{
\begin{tikzpicture}
  \exphdef

  \TState{3}{a_0,b_1,z_0}
\end{tikzpicture}
}\end{center}
}

% 4 : Actions
\only<4->{
\tikzstyle{tick}=[densely dotted]
\tikzstyle{hit}=[->,>=angle 45]
\tikzstyle{selfhit}=[min distance=30pt,curve to]
\tikzstyle{bounce}=[densely dotted,>=stealth',->]
\tikzstyle{hlhit}=[very thick]
\begin{center}\scalebox{\scaleex}{
\begin{tikzpicture}
\exphdef
  \TState{4-5}{a_0,b_1,z_0}
  \TState{6}{a_0,b_1,z_1}
  \TState{7}{a_1,b_1,z_1}
  \TState{8}{a_1,b_1,z_2}

  \only<5>{
    \THit{b_1}{hl}{z_0}{.west}{z_1}
    \path[bounce,bend left,hl] \TBounce{z_0}{}{z_1}{.south};
  }
  \only<6>{
    \THit{a_0}{out=250,in=200,selfhit,hl}{a_0}{.west}{a_1}
    \path[bounce,bend left,hl] \TBounce{a_0}{}{a_1}{.south};
  }
  \only<7>{
    \THit{a_1}{hl}{z_1}{.west}{z_2}
    \path[bounce,bend left,hl] \TBounce{z_1}{}{z_2}{.south};
  }
\end{tikzpicture}
}\end{center}
}

\medskip
\begin{liste}
  \item \tval{Sorts}: components \qex{$a$, $b$, $z$}
\pause[2]
  \item \tval{Processes}: local states / discrete expression levels \qex{$z_0$, $z_1$, $z_2$}
\pause[3]
  \item \tval{States}: sets of active processes%
  \only<3-5>{\qex{$\PHetat{a_0, b_1, z_0}$}}%
  \only<6>{\qex{$\PHetat{a_0, b_1, z_1}$}}%
  \only<7>{\qex{$\PHetat{a_1, b_1, z_1}$}}%
  \only<8>{\qex{$\PHetat{a_1, b_1, z_2}$}}%
\pause[4]
  \item \tval{Actions}: dynamics \qex{\only<5>{\underline}{$\PHfrappe{b_1}{z_0}{z_1}$}, \only<6>{\underline}{$\PHfrappe{a_0}{a_0}{a_1}$}, \only<7>{\underline}{$\PHfrappe{a_1}{z_1}{z_2}$}}
\end{liste}
\end{frame}



\begin{frame}
  \frametitle{Cooperations}
  \framesubtitle{\tcite{\cpmrtcsb}}

\begin{center}\scalebox{\scaleex}{
\begin{tikzpicture}
  \exphcoop
  
  \TState{9}{a_1,b_1}
  \TState{10}{a_1,b_1,ab_0}
  \TState{11}{a_1,b_1,ab_1}
  \TState{12}{a_1,b_1,ab_3}
  
  \node at (a_1.center) {\textasteriskcentered};
  \node at (b_1.center) {\textasteriskcentered};
  \node<5-> at (ab_3.center) {\textasteriskcentered};
  
  \only<9>{
    \node[hl process] at (ab_0.center) {};
    \node[hl process] at (ab_1.center) {};
    \node[hl process] at (ab_2.center) {};
    \node[current process] at (ab_3.center) {};
  }
  
  \only<10>{
    \THit{b_1}{hl}{ab_0}{.210}{ab_1}
    \path[bounce,bend left,hl] \TBounce{ab_0}{}{ab_1}{.240} ;
  }
  
  \only<11>{
    \THit{a_1}{hl}{ab_1}{.160}{ab_3}
    \path[bounce,bend left,hl] \TBounce{ab_1}{}{ab_3}{.south west} ;
  }
\end{tikzpicture}
}\end{center}

\medskip
\begin{liste}
  \item \tval{Cooperation} between \ex{$a_1$} and \ex{$b_1$}: \qex{$\PHfrappe{\underline{a_1 \wedge b_1}}{z_0}{z_1}$}
\pause[5]
  \item Solution: a \tval{cooperative sort} \qex{$ab$} \quad to express \qex{$\underline{a_1 \wedge b_1}$}
\pause[9]
  \item Each configuration is represented by one process \qex{$\underline{a_1 \wedge b_1} \Rightarrow ab_{11}$}
%\pause[15]
%  \item Advantage: regular sort; drawbacks: complexity, temporal shift
\end{liste}
\end{frame}

% Présentation de l'analyse statique

\begin{frame}
  \frametitle{Approximations for the Reachability Analysis}
  \framesubtitle{\tcite{\cpmrmscs}}

Check reachability properties:
\begin{center}
  «~From an initial state $s_0$, is it possible to reach a state $s_n$ where $a_i$ is active?~»
\end{center}
Approximations: $P$ and $Q$, built so that \tval{$P \Rightarrow R \Rightarrow Q$}

\begin{center}
\scalebox{0.6}{
\begin{tikzpicture}
  \path[use as bounding box] (-5,-3.5) rectangle (5,3.5);
  \definecolor{r2}{RGB}{238,10,38}

  %\path<2->[shading=1, inner color=r2, outer color=white] (3.5,-2.8) -- (4.4,3.2) -- (0,3) -- (-4.5,1.4) -- (-2.5,-2.5) -- (0,-3.6) -- (2.8,-2.8);
  \draw<2->[shading=2, inner color=r2, outer color=white, rounded corners, draw=none] (-6,3.5) rectangle (6,-3.5);
  %\draw<2->[thick,fill=white] (2.5,-2.1) -- (3,2.5) -- (-2.7,1.3) -- (-2,-2) -- (2.5,-2.1);
  \draw<2->[thick,fill=white] (-2.8,2) rectangle (2.8,-2);
  %\draw<6->[thick,fill=lightyellow] (2.5,-2.1) -- (3,2.5) -- (-2.7,1.3) -- (-2,-2) -- (2.5,-2.1);
  \draw<6->[thick,fill=lightyellow] (-2.8,2) rectangle (2.8,-2);

  \node<2->[text width=3.5cm, color=red] (s1) at (-5,2) {Over-approximation};
  \path<2->[->,very thick,color=red] (s1.south) edge (-3,1.2);
  \node<2->[text width=3cm,color=black] (q) at (2.2,2.3) {$\neg Q$};

  %\draw<4->[thick, shading=1, top color=darkgreen, bottom color=green] (.5,-.8) -- (1,0) -- (.3,1) -- (-1,.5) -- (-.5,-.5) -- (.5,-.8);
  \draw<4->[thick, shading=1, top color=darkgreen, bottom color=green] (-1.5,.7) rectangle (1.5,-.7);;
  \node<4->[text width=3.5cm,color=darkgreen] (s2) at (5.2,-2.5) {Under-approximation};
  \node<4->[text width=3cm,color=black] (p) at (1.8,.4) {$P$};
  \path<4->[->,very thick,color=darkgreen] (s2) edge (1,-.8);

  \node[text width=3cm,align=center,color=darkcyan] (s) at (0,-1.7) {Exact solution};
  \node<1->[text width=3cm,color=darkcyan] (s0) at (0,0) {};
  \draw[color=darkcyan, ultra thick] (0,0) ellipse (2 and 1.5);
  \node[text width=3cm,color=black] (r) at (2,1.2) {$R$};

  \only<3->{
    \node[point] at (-3.3,-1) {};
    \node[point] at (2,2.5) {};
  }
  \only<5->{
    \node[point] at (-1,.2) {};
    \node[point] at (1.1,-.5) {};
  }
  \only<7->{
    \node[point] at (-.5,-1.1) {};
    \node[point] at (2.5,1) {};
  }
\end{tikzpicture}
}
\end{center}

\uncover<8->{
Polynomial complexity in the number of sorts\\
Exponential complexity in the number of processes in each sort
\begin{fleches}
  \item Efficient for big models with few expression levels
\end{fleches}
}
\end{frame}


\section{Enriching the Process Hitting}
% Présentation de Metazoan

\begin{frame}[c]
  \frametitle{Standard Process Hitting}

\begin{tikzpicture}
  \path[use as bounding box] (-5.2,-4) rectangle (5.2,3.5);
  \planPHstandard
  \planPHp[stillhidden]
  \planPHan[stillhidden]
  \planPHmult[stillhidden]
  \planPHcanonique[stillhidden]
\end{tikzpicture}

\end{frame}



\begin{frame}[t]
  \frametitle{Permissiveness of the Standard Dynamics}
  \framesubtitle{Model extracted from \tcite{François \textit{et al.} in Molecular Systems Biology, 2007}}

\makenoprio

\begin{tikzpicture}
  \path[use as bounding box] (-1,0) rectangle (8,6.5);
  
  \draw<2>[very thick, darkgreen, fill=green!10] (7,4.5) ellipse [x radius=1cm,y radius=2cm];
  \node<2>[darkgreen, align=flush left, text width=3cm] at (7.7,2) {\Large \tval{Pigment productions}};
  \draw<2> (8,4) node[anchor=west, darkgreen] {0 = inactive}
           (8,5) node[anchor=west, darkgreen] {1 = production};
  \draw<3-> (7.75,4.02) node[anchor=west, darkgreen] {= inactive}
           (7.75,5) node[anchor=west, darkgreen] {= production};
  
  \draw<3>[very thick, darkred, fill=red!10] (0,4.7) ellipse [x radius=1cm,y radius=1.5cm];
  \node<3>[darkred, align=flush left, text width=2cm, anchor=west] at (1,5.8) {\Large \tval{Clock}};
  
  \draw<4>[very thick, darkviolet, fill=violet!10] (1,0.5) ellipse [x radius=1cm,y radius=1.5cm];
  \node<4>[darkviolet, align=flush left, anchor=west] at (2,-.3) {\Large \tval{Wavefront progression}};
  \draw<4> (0,0) node[anchor=east, darkviolet] {0 = “Off”}
           (0,1) node[anchor=east, darkviolet] {1 = “On”};
  
  \exmetazoan
  
%   \only<-4>{
%     \THit{f_1.north east}{selfhit, min distance=30, bend left, out=150, in=90}{f_1}{.south east}{f_0}
%     \path[bounce, bend left=50]
%       \TBounce{f_1}{}{f_0}{.north east};
% %   }
%   
% %   \only<5>{
% %     \THit{f_1.north east}{selfhit, min distance=30, bend left, out=150, in=90,hlr}{f_1}{.south east}{f_0}
% %     \path[bounce, bend left=50]
% %       \TBounce{f_1}{hlr}{f_0}{.north east};
% %   }
%   
% %   \only<-5>{
%     \THit{f_0.east}{bend right=60, in=-140}{c_1}{.south east}{c_0}
%     \path[bounce, bend left=50]
%       \TBounce{c_1}{}{c_0}{.north east};
%   }
  
%   \only<6>{
%     \THit{f_0.east}{bend right=60, in=-140, hlr}{c_1}{.south east}{c_0}
%     \path[bounce, bend left=50]
%       \TBounce{c_1}{hlr}{c_0}{.north east};
%   }
  
  \only<7,13,26,28,30>{
    \THit{fc_2}{\prio, hl}{a_0}{.west}{a_1} \path[bounce, bend left=50, hl] \TBounce{a_0}{\prio}{a_1}{.south west};
  }
  \only<8,14,21,23,25,32>{
    \THit{f_1}{bend left=30, in=90, hl}{c_0}{.west}{c_1} \path[bounce, bend left=50, hl] \TBounce{c_0}{}{c_1}{.south west};
  }
  \only<9,12,15,18>{
    \path (0.8, 4.5) edge[\superprio,coopupdate,hl] (3.2, 3);
  }
  \only<10,16,27,29>{
    \THit{c_1}{\prio, hl}{a_1}{.west}{a_0} \path[bounce, bend right=50, hl] \TBounce{a_1}{\prio}{a_0}{.north west};
  }
  \only<11,17,22,24,31>{
    \THit{c_1}{selfhit, hl}{c_1}{.west}{c_0} \path[bounce, bend right=50, hl] \TBounce{c_1}{}{c_0}{.north west};
  }
  \only<19>{
    \THit{f_1.north east}{selfhit, min distance=30, bend left, out=150, in=90, hl}{f_1}{.south east}{f_0} \path[bounce, bend left=50, hl] \TBounce{f_1}{}{f_0}{.north east};
  }
  
  \TState{6-7,13,19,21,23}{f_1, a_0, c_0, fc_2}
  \TState{8,14}{f_1, a_1, c_0, fc_2}
  \TState{9,15}{f_1, a_1, c_1, fc_2}
  \TState{10,16}{f_1, a_1, c_1, fc_3}
  \TState{11,17}{f_1, a_0, c_1, fc_3}
  \TState{12,18}{f_1, a_0, c_0, fc_3}
  \TState{20}{f_0, a_0, c_0, fc_2}

  \TState{22,24,26}{f_1, a_0, c_1, fc_2}
  \TState{25}{f_1, a_0, c_0, fc_2}

  \TState{27,29,31,33}{f_1, a_1, c_1, fc_2}
  \TState{28,30}{f_1, a_0, c_1, fc_2}
  \TState{32}{f_1, a_1, c_0, fc_2}
\end{tikzpicture}

\pause[7]

\vspace*{-3cm}
\hfill
\begin{tikzpicture}

\tikz \foreach \x in {0,...,12}
  \draw[dotted] (\x/4,0) -- (\x/4,1.5);

\draw[dotted] (0,0) -- (-3,0);
\draw[dotted] (0,1.5) -- (-3,1.5);

\only<8->{\fill (-3,0) rectangle (-2.75,1.5);}
\only<9->{\fill (-2.75,0) rectangle (-2.5,1.5);}
\only<10->{\fill (-2.5,0) rectangle (-2.25,1.5);}
\only<11->{\fill[gray!30] (-2.25,0) rectangle (-2,1.5);}
\only<12->{\fill[gray!30] (-2,0) rectangle (-1.75,1.5);}
\only<13->{\fill[gray!30] (-1.75,0) rectangle (-1.5,1.5);}
\only<14->{\fill (-1.5,0) rectangle (-1.25,1.5);}
\only<15->{\fill (-1.25,0) rectangle (-1,1.5);}
\only<16->{\fill (-1,0) rectangle (-0.75,1.5);}
\only<17->{\fill[gray!30] (-0.75,0) rectangle (-0.5,1.5);}
\only<18->{\fill[gray!30] (-0.5,0) rectangle (-0.25,1.5);}
\only<19->{\fill[gray!30] (-0.25,0) rectangle (0,1.5);}

\end{tikzpicture}

\pause[21]
\vspace*{.5cm}
\hfill
\begin{tikzpicture}

\tikz \foreach \x in {0,...,12}
  \draw[dotted] (\x/4,0) -- (\x/4,1.5);

\draw[dotted] (0,0) -- (-3,0);
\draw[dotted] (0,1.5) -- (-3,1.5);

\only<22->{\fill[gray!30] (-3,0) rectangle (-2.75,1.5);}
\only<23->{\fill[gray!30] (-2.75,0) rectangle (-2.5,1.5);}
\only<24->{\fill[gray!30] (-2.5,0) rectangle (-2.25,1.5);}
\only<25->{\fill[gray!30] (-2.25,0) rectangle (-2,1.5);}
\only<26->{\fill[gray!30] (-2,0) rectangle (-1.75,1.5);}
\only<27->{\fill (-1.75,0) rectangle (-1.5,1.5);}
\only<28->{\fill[gray!30] (-1.5,0) rectangle (-1.25,1.5);}
\only<29->{\fill (-1.25,0) rectangle (-1,1.5);}
\only<30->{\fill[gray!30] (-1,0) rectangle (-0.75,1.5);}
\only<31->{\fill (-0.75,0) rectangle (-0.5,1.5);}
\only<32->{\fill (-0.5,0) rectangle (-0.25,1.5);}
\only<33->{\fill (-0.25,0) rectangle (0,1.5);}

\end{tikzpicture}

\pause[15]

\end{frame}

\subsection{Classes of Priorities}
% Définition des priorités

\begin{frame}[c]
  \frametitle{Process Hitting with Classes of Priorities}

\begin{tikzpicture}
  \path[use as bounding box] (-5.2,-4) rectangle (5.2,3.5);
  \planPHstandard
  \planPHp
  \planPHan[stillhidden]
  \planPHmult[stillhidden]
  \planPHcanonique[stillhidden]
\end{tikzpicture}

\end{frame}



\begin{frame}[t]
  \frametitle{Addition of classes of priorities}
  \framesubtitle{\tcite{\cfpmrcsbio}}

\bigskip
\begin{itemize}
  \item Each action is associated to a discrete priority
  \item An action is playable only if no other action with higher priority is playable
\end{itemize}

\medskip

\begin{center}
% \begin{tabular}{ccccc}
%   \hspace*{.3cm}\tikz \node[labelprio1] {$1$}; \hspace*{.3cm} &
%   \hspace*{.3cm}\tikz \node[labelprio2] {$2$}; \hspace*{.3cm} &
%   \hspace*{.3cm}\tikz \node[labelprio3] {$3$}; \hspace*{.3cm} &
%   \vspace*{.5em}\hspace*{.3cm}\raisebox{5pt}{\ldots}\hspace*{.3cm} &
%   \hspace*{.3cm}\tikz \node[labelprion] {$n$}; \hspace*{.3cm} \\\hline
%   \multicolumn{2}{l}{
%   \parbox{1.5cm}{\vspace*{.5em}plus haute\\priorité}} &
%   %\parbox{1cm}{~} &
%   \parbox{1cm}{~} &&
%   \parbox{1.5cm}{\vspace*{.5em}plus basse\\priorité}
% \end{tabular}
% \hspace*{-1em}
% \raisebox{2.2pt}{$\blacktriangleright$}

\begin{tabular}{*{5}{>{\centering}p{1cm}}}
  \tikz \node[labelprio1] {$1$}; &
  \tikz \node[labelprio2] {$2$}; &
  \tikz \node[labelprio3] {$3$}; &
  \raisebox{5pt}{\ldots} &
  \tikz \node[labelprion] {$n$};
\vspace*{.5em} \tabularnewline \hline
  \multicolumn{2}{l}{\parbox{1.5cm}{\vspace*{.5em}highest\\priority}} &&
  \multicolumn{2}{r}{\parbox{1.5cm}{\raggedleft\vspace*{.5em}lowest\\priority}}
\end{tabular}
\hspace*{-1em}
\raisebox{2.2pt}{$\blacktriangleright$}

\bigskip

% \only<2-3>{
\bigskip
\begin{tikzpicture}
  \path[use as bounding box] (-0.5,-0.5) rectangle (2.5,1.5);
  \TSort{(0,0)}{a}{2}{l}
  \TSort{(2,0)}{b}{2}{r}
  \THit{a_0}{}{b_0}{.west}{b_1}
  \THit{a_0}{out=-120,in=180,selfhit}{a_0}{.west}{a_1}
  \path[bounce]
  \TBounce{a_0}{bend left}{a_1}{.south}
  \TBounce{b_0}{bend left}{b_1}{.south}
  ;
  \TState{-1}{a_0,b_0}
  \TState{2-}{a_1,b_0}

  \node[labelprio1] at (-1.5,-0.5) {$1$};
  \node[labelprio2] at (1,0.25) {$2$};
\end{tikzpicture}

\bigskip

\f $b_1$ \tval{cannot} be reached
% }
\end{center}

% \only<4->{
% \begin{itemize}
%   \item Allow to model classes of actions with similar speeds or temporal parameters
% \end{itemize}
% \begin{center}
% \begin{tabular}{*{5}{>{\centering}p{1cm}}}
%   \tikz \node[labelprio1,labelstocha] {$A$}; &
%   \tikz \node[labelprio2,labelstocha] {$B$}; &
%   \tikz \node[labelprio3,labelstocha] {$C$}; &
%   \raisebox{5pt}{\ldots} &
%   \tikz \node[labelprion,labelstocha] {$N$};
% \vspace*{.5em} \tabularnewline \hline
%   \multicolumn{1}{r}{\parbox{1cm}{\hspace*{-1.7cm}\parbox{2.5cm}{\raggedleft\vspace*{.5em}\tval{instantaneous}}}} &
%   \multicolumn{2}{l}{\parbox{2cm}{\vspace*{.5em}\tval{very fast}}} &
%   \multicolumn{2}{r}{\parbox{2cm}{\raggedleft\vspace*{.5em}\tval{very slow}}}
% \end{tabular}
% \hspace*{-1em}
% \raisebox{-.4em}{$\blacktriangleright$}
% \end{center}
% }

\end{frame}

% Priorités dans Metazoan

\begin{frame}[t]
  \frametitle{Use of the Classes of Priorities}
  \framesubtitle{\tcite{\cfpmrcsbio}}

\makenoprio

\begin{tikzpicture}
  \path[use as bounding box] (-2,0) rectangle (8,6.5);
  \exmetazoan

  \node[labelprio1] at (2.3,4) {$1$};
  \node[labelprio1] at (2.6,0.8) {$1$};
  \node[labelprio2] at (5.5,3.9) {$2$};
  \node[labelprio2] at (3.5,5.3) {$2$};
  \node[labelprio3] at (0,2.5) {$3$};
  \node[labelprio3] at (0.8,5.8) {$3$};
  %\node[labelprio4] at (1.5,1.8) {$4$};
  
  \TState{3,9,15}{f_1, a_0, c_0, fc_2}
  \TState{4,10}{f_1, a_1, c_0, fc_2}
  \TState{5,11}{f_1, a_1, c_1, fc_2}
  \TState{6,12}{f_1, a_1, c_1, fc_3}
  \TState{7,13}{f_1, a_0, c_1, fc_3}
  \TState{8,14}{f_1, a_0, c_0, fc_3}
\end{tikzpicture}

\pause[2]
\vspace*{-2.5cm}
\hfill
\begin{tikzpicture}
  \tikz \foreach \x in {0,...,12}
    \draw[dotted] (\x/4,0) -- (\x/4,1.5);

  \draw[dotted] (0,0) -- (-3,0);
  \draw[dotted] (0,1.5) -- (-3,1.5);

  \only<4->{\fill (-3,0) rectangle (-2.75,1.5);}
  \only<5->{\fill (-2.75,0) rectangle (-2.5,1.5);}
  \only<6->{\fill (-2.5,0) rectangle (-2.25,1.5);}
  \only<7->{\fill[gray!30] (-2.25,0) rectangle (-2,1.5);}
  \only<8->{\fill[gray!30] (-2,0) rectangle (-1.75,1.5);}
  \only<9->{\fill[gray!30] (-1.75,0) rectangle (-1.5,1.5);}
  \only<10->{\fill (-1.5,0) rectangle (-1.25,1.5);}
  \only<11->{\fill (-1.25,0) rectangle (-1,1.5);}
  \only<12->{\fill (-1,0) rectangle (-0.75,1.5);}
  \only<13->{\fill[gray!30] (-0.75,0) rectangle (-0.5,1.5);}
  \only<14->{\fill[gray!30] (-0.5,0) rectangle (-0.25,1.5);}
  \only<15->{\fill[gray!30] (-0.25,0) rectangle (0,1.5);}
\end{tikzpicture}

\pause[15]
%\vspace*{.3cm}
\bigskip
\begin{flushright}
  \f Only one possible stationary behavior
\end{flushright}

\end{frame}



\begin{frame}[c]
  \frametitle{Abstraction of Temporal Parameters}
  \framesubtitle{\tcite{\paulevephd}}

\begin{itemize}
  \item Simulation with stochastic parameters:
\end{itemize}

\medskip

\scalebox{.8}{% GNUPLOT: LaTeX picture
\setlength{\unitlength}{0.240900pt}
\ifx\plotpoint\undefined\newsavebox{\plotpoint}\fi
\sbox{\plotpoint}{\rule[-0.200pt]{0.400pt}{0.400pt}}%
\begin{picture}(1440,188)(0,0)
\font\gnuplot=cmtt10 at 8pt
\gnuplot
\put(20,111){\makebox(0,0){$a$}}
\sbox{\plotpoint}{\rule[-0.200pt]{0.400pt}{0.400pt}}%
\put(80.0,66.0){\rule[-0.200pt]{4.818pt}{0.400pt}}
\put(64,66){\makebox(0,0)[r]{ 0}}
\put(1371.0,66.0){\rule[-0.200pt]{4.818pt}{0.400pt}}
\put(80.0,157.0){\rule[-0.200pt]{4.818pt}{0.400pt}}
\put(64,157){\makebox(0,0)[r]{ 1}}
\put(1371.0,157.0){\rule[-0.200pt]{4.818pt}{0.400pt}}
\put(80.0,66.0){\rule[-0.200pt]{0.400pt}{4.818pt}}
\put(80,33){\makebox(0,0){ 0}}
\put(80.0,137.0){\rule[-0.200pt]{0.400pt}{4.818pt}}
\put(244.0,66.0){\rule[-0.200pt]{0.400pt}{4.818pt}}
\put(244,33){\makebox(0,0){ 5}}
\put(244.0,137.0){\rule[-0.200pt]{0.400pt}{4.818pt}}
\put(408.0,66.0){\rule[-0.200pt]{0.400pt}{4.818pt}}
\put(408,33){\makebox(0,0){ 10}}
\put(408.0,137.0){\rule[-0.200pt]{0.400pt}{4.818pt}}
\put(572.0,66.0){\rule[-0.200pt]{0.400pt}{4.818pt}}
\put(572,33){\makebox(0,0){ 15}}
\put(572.0,137.0){\rule[-0.200pt]{0.400pt}{4.818pt}}
\put(735.0,66.0){\rule[-0.200pt]{0.400pt}{4.818pt}}
\put(735,33){\makebox(0,0){ 20}}
\put(735.0,137.0){\rule[-0.200pt]{0.400pt}{4.818pt}}
\put(899.0,66.0){\rule[-0.200pt]{0.400pt}{4.818pt}}
\put(899,33){\makebox(0,0){ 25}}
\put(899.0,137.0){\rule[-0.200pt]{0.400pt}{4.818pt}}
\put(1063.0,66.0){\rule[-0.200pt]{0.400pt}{4.818pt}}
\put(1063,33){\makebox(0,0){ 30}}
\put(1063.0,137.0){\rule[-0.200pt]{0.400pt}{4.818pt}}
\put(1227.0,66.0){\rule[-0.200pt]{0.400pt}{4.818pt}}
\put(1227,33){\makebox(0,0){ 35}}
\put(1227.0,137.0){\rule[-0.200pt]{0.400pt}{4.818pt}}
\put(1391.0,66.0){\rule[-0.200pt]{0.400pt}{4.818pt}}
\put(1391,33){\makebox(0,0){ 40}}
\put(1391.0,137.0){\rule[-0.200pt]{0.400pt}{4.818pt}}
\put(80,66){\usebox{\plotpoint}}
\put(80.0,66.0){\rule[-0.200pt]{9.636pt}{0.400pt}}
\put(120.0,66.0){\rule[-0.200pt]{0.400pt}{21.922pt}}
\put(120.0,157.0){\rule[-0.200pt]{14.213pt}{0.400pt}}
\put(179.0,66.0){\rule[-0.200pt]{0.400pt}{21.922pt}}
\put(179.0,66.0){\rule[-0.200pt]{15.899pt}{0.400pt}}
\put(245.0,66.0){\rule[-0.200pt]{0.400pt}{21.922pt}}
\put(245.0,157.0){\rule[-0.200pt]{12.045pt}{0.400pt}}
\put(295.0,66.0){\rule[-0.200pt]{0.400pt}{21.922pt}}
\put(295.0,66.0){\rule[-0.200pt]{19.272pt}{0.400pt}}
\put(375.0,66.0){\rule[-0.200pt]{0.400pt}{21.922pt}}
\put(375.0,157.0){\rule[-0.200pt]{12.045pt}{0.400pt}}
\put(425.0,66.0){\rule[-0.200pt]{0.400pt}{21.922pt}}
\put(425.0,66.0){\rule[-0.200pt]{18.549pt}{0.400pt}}
\put(502.0,66.0){\rule[-0.200pt]{0.400pt}{21.922pt}}
\put(502.0,157.0){\rule[-0.200pt]{15.658pt}{0.400pt}}
\put(567.0,66.0){\rule[-0.200pt]{0.400pt}{21.922pt}}
\put(567.0,66.0){\rule[-0.200pt]{10.600pt}{0.400pt}}
\put(611.0,66.0){\rule[-0.200pt]{0.400pt}{21.922pt}}
\put(611.0,157.0){\rule[-0.200pt]{13.972pt}{0.400pt}}
\put(669.0,66.0){\rule[-0.200pt]{0.400pt}{21.922pt}}
\put(669.0,66.0){\rule[-0.200pt]{17.345pt}{0.400pt}}
\put(741.0,66.0){\rule[-0.200pt]{0.400pt}{21.922pt}}
\put(741.0,157.0){\rule[-0.200pt]{17.586pt}{0.400pt}}
\put(814.0,66.0){\rule[-0.200pt]{0.400pt}{21.922pt}}
\put(814.0,66.0){\rule[-0.200pt]{15.658pt}{0.400pt}}
\put(879.0,66.0){\rule[-0.200pt]{0.400pt}{21.922pt}}
\put(879.0,157.0){\rule[-0.200pt]{18.549pt}{0.400pt}}
\put(956.0,66.0){\rule[-0.200pt]{0.400pt}{21.922pt}}
\put(956.0,66.0){\rule[-0.200pt]{16.622pt}{0.400pt}}
\put(1025.0,66.0){\rule[-0.200pt]{0.400pt}{21.922pt}}
\put(1025.0,157.0){\rule[-0.200pt]{88.169pt}{0.400pt}}
\end{picture}
}
\scalebox{.8}{% GNUPLOT: LaTeX picture
\setlength{\unitlength}{0.240900pt}
\ifx\plotpoint\undefined\newsavebox{\plotpoint}\fi
\begin{picture}(1440,188)(0,0)
\font\gnuplot=cmtt10 at 8pt
\gnuplot
\put(20,111){\makebox(0,0){$c$}}
\sbox{\plotpoint}{\rule[-0.200pt]{0.400pt}{0.400pt}}%
\put(80.0,66.0){\rule[-0.200pt]{4.818pt}{0.400pt}}
\put(64,66){\makebox(0,0)[r]{ 0}}
\put(1371.0,66.0){\rule[-0.200pt]{4.818pt}{0.400pt}}
\put(80.0,157.0){\rule[-0.200pt]{4.818pt}{0.400pt}}
\put(64,157){\makebox(0,0)[r]{ 1}}
\put(1371.0,157.0){\rule[-0.200pt]{4.818pt}{0.400pt}}
\put(80.0,66.0){\rule[-0.200pt]{0.400pt}{4.818pt}}
\put(80,33){\makebox(0,0){ 0}}
\put(80.0,137.0){\rule[-0.200pt]{0.400pt}{4.818pt}}
\put(244.0,66.0){\rule[-0.200pt]{0.400pt}{4.818pt}}
\put(244,33){\makebox(0,0){ 5}}
\put(244.0,137.0){\rule[-0.200pt]{0.400pt}{4.818pt}}
\put(408.0,66.0){\rule[-0.200pt]{0.400pt}{4.818pt}}
\put(408,33){\makebox(0,0){ 10}}
\put(408.0,137.0){\rule[-0.200pt]{0.400pt}{4.818pt}}
\put(572.0,66.0){\rule[-0.200pt]{0.400pt}{4.818pt}}
\put(572,33){\makebox(0,0){ 15}}
\put(572.0,137.0){\rule[-0.200pt]{0.400pt}{4.818pt}}
\put(735.0,66.0){\rule[-0.200pt]{0.400pt}{4.818pt}}
\put(735,33){\makebox(0,0){ 20}}
\put(735.0,137.0){\rule[-0.200pt]{0.400pt}{4.818pt}}
\put(899.0,66.0){\rule[-0.200pt]{0.400pt}{4.818pt}}
\put(899,33){\makebox(0,0){ 25}}
\put(899.0,137.0){\rule[-0.200pt]{0.400pt}{4.818pt}}
\put(1063.0,66.0){\rule[-0.200pt]{0.400pt}{4.818pt}}
\put(1063,33){\makebox(0,0){ 30}}
\put(1063.0,137.0){\rule[-0.200pt]{0.400pt}{4.818pt}}
\put(1227.0,66.0){\rule[-0.200pt]{0.400pt}{4.818pt}}
\put(1227,33){\makebox(0,0){ 35}}
\put(1227.0,137.0){\rule[-0.200pt]{0.400pt}{4.818pt}}
\put(1391.0,66.0){\rule[-0.200pt]{0.400pt}{4.818pt}}
\put(1391,33){\makebox(0,0){ 40}}
\put(1391.0,137.0){\rule[-0.200pt]{0.400pt}{4.818pt}}
\put(80,66){\usebox{\plotpoint}}
\put(80.0,66.0){\rule[-0.200pt]{16.622pt}{0.400pt}}
\put(149.0,66.0){\rule[-0.200pt]{0.400pt}{21.922pt}}
\put(149.0,157.0){\rule[-0.200pt]{14.454pt}{0.400pt}}
\put(209.0,66.0){\rule[-0.200pt]{0.400pt}{21.922pt}}
\put(209.0,66.0){\rule[-0.200pt]{13.490pt}{0.400pt}}
\put(265.0,66.0){\rule[-0.200pt]{0.400pt}{21.922pt}}
\put(265.0,157.0){\rule[-0.200pt]{14.936pt}{0.400pt}}
\put(327.0,66.0){\rule[-0.200pt]{0.400pt}{21.922pt}}
\put(327.0,66.0){\rule[-0.200pt]{16.140pt}{0.400pt}}
\put(394.0,66.0){\rule[-0.200pt]{0.400pt}{21.922pt}}
\put(394.0,157.0){\rule[-0.200pt]{18.308pt}{0.400pt}}
\put(470.0,66.0){\rule[-0.200pt]{0.400pt}{21.922pt}}
\put(470.0,66.0){\rule[-0.200pt]{14.936pt}{0.400pt}}
\put(532.0,66.0){\rule[-0.200pt]{0.400pt}{21.922pt}}
\put(532.0,157.0){\rule[-0.200pt]{12.045pt}{0.400pt}}
\put(582.0,66.0){\rule[-0.200pt]{0.400pt}{21.922pt}}
\put(582.0,66.0){\rule[-0.200pt]{13.731pt}{0.400pt}}
\put(639.0,66.0){\rule[-0.200pt]{0.400pt}{21.922pt}}
\put(639.0,157.0){\rule[-0.200pt]{16.622pt}{0.400pt}}
\put(708.0,66.0){\rule[-0.200pt]{0.400pt}{21.922pt}}
\put(708.0,66.0){\rule[-0.200pt]{17.345pt}{0.400pt}}
\put(780.0,66.0){\rule[-0.200pt]{0.400pt}{21.922pt}}
\put(780.0,157.0){\rule[-0.200pt]{14.936pt}{0.400pt}}
\put(842.0,66.0){\rule[-0.200pt]{0.400pt}{21.922pt}}
\put(842.0,66.0){\rule[-0.200pt]{19.995pt}{0.400pt}}
\put(925.0,66.0){\rule[-0.200pt]{0.400pt}{21.922pt}}
\put(925.0,157.0){\rule[-0.200pt]{15.899pt}{0.400pt}}
\put(991.0,66.0){\rule[-0.200pt]{0.400pt}{21.922pt}}
\put(991.0,66.0){\rule[-0.200pt]{17.345pt}{0.400pt}}
\put(1063.0,66.0){\rule[-0.200pt]{0.400pt}{21.922pt}}
\put(1063.0,157.0){\rule[-0.200pt]{4.336pt}{0.400pt}}
\put(1081.0,66.0){\rule[-0.200pt]{0.400pt}{21.922pt}}
\put(1081.0,66.0){\rule[-0.200pt]{74.679pt}{0.400pt}}
\end{picture}
}
\scalebox{.8}{% GNUPLOT: LaTeX picture
\setlength{\unitlength}{0.240900pt}
\ifx\plotpoint\undefined\newsavebox{\plotpoint}\fi
\begin{picture}(1440,188)(0,0)
\font\gnuplot=cmtt10 at 8pt
\gnuplot
\put(20,111){\makebox(0,0){$f$}}
\sbox{\plotpoint}{\rule[-0.200pt]{0.400pt}{0.400pt}}%
\put(80.0,66.0){\rule[-0.200pt]{4.818pt}{0.400pt}}
\put(64,66){\makebox(0,0)[r]{ 0}}
\put(1371.0,66.0){\rule[-0.200pt]{4.818pt}{0.400pt}}
\put(80.0,157.0){\rule[-0.200pt]{4.818pt}{0.400pt}}
\put(64,157){\makebox(0,0)[r]{ 1}}
\put(1371.0,157.0){\rule[-0.200pt]{4.818pt}{0.400pt}}
\put(80.0,66.0){\rule[-0.200pt]{0.400pt}{4.818pt}}
\put(80,33){\makebox(0,0){ 0}}
\put(80.0,137.0){\rule[-0.200pt]{0.400pt}{4.818pt}}
\put(244.0,66.0){\rule[-0.200pt]{0.400pt}{4.818pt}}
\put(244,33){\makebox(0,0){ 5}}
\put(244.0,137.0){\rule[-0.200pt]{0.400pt}{4.818pt}}
\put(408.0,66.0){\rule[-0.200pt]{0.400pt}{4.818pt}}
\put(408,33){\makebox(0,0){ 10}}
\put(408.0,137.0){\rule[-0.200pt]{0.400pt}{4.818pt}}
\put(572.0,66.0){\rule[-0.200pt]{0.400pt}{4.818pt}}
\put(572,33){\makebox(0,0){ 15}}
\put(572.0,137.0){\rule[-0.200pt]{0.400pt}{4.818pt}}
\put(735.0,66.0){\rule[-0.200pt]{0.400pt}{4.818pt}}
\put(735,33){\makebox(0,0){ 20}}
\put(735.0,137.0){\rule[-0.200pt]{0.400pt}{4.818pt}}
\put(899.0,66.0){\rule[-0.200pt]{0.400pt}{4.818pt}}
\put(899,33){\makebox(0,0){ 25}}
\put(899.0,137.0){\rule[-0.200pt]{0.400pt}{4.818pt}}
\put(1063.0,66.0){\rule[-0.200pt]{0.400pt}{4.818pt}}
\put(1063,33){\makebox(0,0){ 30}}
\put(1063.0,137.0){\rule[-0.200pt]{0.400pt}{4.818pt}}
\put(1227.0,66.0){\rule[-0.200pt]{0.400pt}{4.818pt}}
\put(1227,33){\makebox(0,0){ 35}}
\put(1227.0,137.0){\rule[-0.200pt]{0.400pt}{4.818pt}}
\put(1391.0,66.0){\rule[-0.200pt]{0.400pt}{4.818pt}}
\put(1391,33){\makebox(0,0){ 40}}
\put(1391.0,137.0){\rule[-0.200pt]{0.400pt}{4.818pt}}
\put(80,157){\usebox{\plotpoint}}
\put(80.0,157.0){\rule[-0.200pt]{235.841pt}{0.400pt}}
\put(1059.0,66.0){\rule[-0.200pt]{0.400pt}{21.922pt}}
\put(1059.0,66.0){\rule[-0.200pt]{79.979pt}{0.400pt}}
\end{picture}
}

\bigskip
\bigskip

\begin{itemize}
  \item Other possible analysis: stochastic model checkers (PRISM)
  \begin{itemize}
    \item[\f] But combinatoric explosion: PRISM fails for more than 5 components
  \end{itemize}

\end{itemize}

\end{frame}



\begin{frame}[t]
  \frametitle{Addition of classes of priorities}
  \framesubtitle{\tcite{\cfpmrcsbio}}

\bigskip
\begin{itemize}
  \item Each action is associated to a discrete priority
  \item An action is playable only if no other action with higher priority is playable
\end{itemize}

\medskip

\begin{center}

\begin{tabular}{*{5}{>{\centering}p{1cm}}}
  \tikz \node[labelprio1] {$1$}; &
  \tikz \node[labelprio2] {$2$}; &
  \tikz \node[labelprio3] {$3$}; &
  \raisebox{5pt}{\ldots} &
  \tikz \node[labelprion] {$n$};
\vspace*{.5em} \tabularnewline \hline
  \multicolumn{2}{l}{\parbox{1.5cm}{\vspace*{.5em}highest\\priority}} &&
  \multicolumn{2}{r}{\parbox{1.5cm}{\raggedleft\vspace*{.5em}lowest\\priority}}
\end{tabular}
\hspace*{-1em}
\raisebox{2.2pt}{$\blacktriangleright$}

\bigskip
\end{center}

\begin{itemize}
  \item Allow to model classes of actions with similar speeds or temporal parameters
\end{itemize}
\begin{center}
\begin{tabular}{*{5}{>{\centering}p{1cm}}}
  \tikz \node[labelprio1,labelstocha] {$A$}; &
  \tikz \node[labelprio2,labelstocha] {$B$}; &
  \tikz \node[labelprio3,labelstocha] {$C$}; &
  \raisebox{5pt}{\ldots} &
  \tikz \node[labelprion,labelstocha] {$N$};
\vspace*{.5em} \tabularnewline \hline
  \multicolumn{1}{r}{\parbox{1cm}{\hspace*{-1.7cm}\parbox{2.5cm}{\raggedleft\vspace*{.5em}\tval{instantaneous}}}} &
  \multicolumn{2}{l}{\parbox{2cm}{\vspace*{.5em}\tval{very fast}}} &
  \multicolumn{2}{r}{\parbox{2cm}{\raggedleft\vspace*{.5em}\tval{very slow}}}
\end{tabular}
\hspace*{-1em}
\raisebox{-.4em}{$\blacktriangleright$}
\end{center}

\end{frame}



\begin{frame}[t]
  \frametitle{Limitation of the Classes of Priorities}
  \framesubtitle{\tcite{\cfpmrcsbio}}

\makenoprio

\begin{tikzpicture}
  \path[use as bounding box] (-2,0) rectangle (8,6.5);
  \exmetazoan
  
  \only<2>{
    \THit{f_1.north east}{selfhit, min distance=30, bend left, out=150, in=90,hlr}{f_1}{.south east}{f_0}
    \path[bounce, bend left=50]
      \TBounce{f_1}{hlr}{f_0}{.north east};
  }
  
  \node[labelprio1] at (2.3,4) {$1$};
  \node[labelprio1] at (2.6,0.8) {$1$};
  \node[labelprio2] at (5.5,3.9) {$2$};
  \node[labelprio2] at (3.5,5.3) {$2$};
  \node[labelprio3] at (0,2.5) {$3$};
  \node[labelprio3] at (0.8,5.8) {$3$};
  
  \node[labelprio3] at (2.2,2.5) {$3$};
  \node[labelprio4] at (1.5,1.8) {$4$};
  \node<3->[labelprio3] at (1.55,1.8) {$3$};
  
%   \TState{3,9,15}{f_1, a_0, c_0, fc_2}
%   \TState{4,10}{f_1, a_1, c_0, fc_2}
%   \TState{5,11}{f_1, a_1, c_1, fc_2}
%   \TState{6,12}{f_1, a_1, c_1, fc_3}
%   \TState{7,13}{f_1, a_0, c_1, fc_3}
%   \TState{8,14}{f_1, a_0, c_0, fc_3}
\end{tikzpicture}

\pause[2]
\vspace*{-1cm}
%\vspace*{.3cm}
\bigskip
\begin{flushright}
  \f Memoryless (no accumulation)\\
  Unplayable action: $\PHfrappe{f_1}{f_1}{f_0}$
\end{flushright}

\end{frame}

\subsection{Neutralizing Edges}
% Arcs neutralisants

\begin{frame}[c]
  \frametitle{Process Hitting with Neutralizing Edges}

\begin{tikzpicture}
  \path[use as bounding box] (-5.2,-4) rectangle (5.2,3.5);
  \planPHstandard
  \planPHp
  \planPHan
  \planPHmult[stillhidden]
  \planPHcanonique[stillhidden]
\end{tikzpicture}

\end{frame}



\begin{frame}[c]
  \frametitle{Addition of Neutralizing Edges}

\begin{columns}
\begin{column}{.4\textwidth}

\begin{tikzpicture}
  %\path[use as bounding box] (-2,0) rectangle (8,6.5);
  \TSort{(0,0)}{a}{2}{l}
  \TSort{(2,0)}{b}{2}{r}
  \TSort{(0,3)}{c}{2}{l}
  \TSort{(2,3)}{d}{2}{r}
  
  \THit{a_0}{}{b_0}{.west}{b_1}
  \path[bounce] \TBounce{b_0}{bend left}{b_1}{.south};
  
  \THit{c_0}{}{d_0}{.west}{d_1}
  \path[bounce] \TBounce{d_0}{bend left}{d_1}{.south};
  
  \node (nea1) at (1,0) {};
  \node[dotne] (nea2) at (1,2.9) {};
  \draw[linene] (nea1) to[out=100, in=-100] (nea2);
  
  \TState{1}{a_0, b_0, c_0, d_0}
  \TState{2}{a_0, b_1, c_0, d_0}
  \TState{3}{a_0, b_1, c_0, d_1}
\end{tikzpicture}

\end{column}
\begin{column}{.55\textwidth}
\begin{center}

\begin{itemize}
  \item Integration of temporal data\\
    about relative reaction speeds
  \item Atomistic preemptions between actions\\
    similar to ``atomistic priorities''
\end{itemize}

\vspace*{1cm}
$\PHfrappe{c_0}{d_0}{d_1}$ cannot be plays \tval{while}

\bigskip
$\PHfrappe{a_0}{b_0}{b_1}$ is playable

\bigskip
\f $d_1$ is \tval{always} reached after $b_1$

\end{center}
\end{column}
\end{columns}

\end{frame}

% Arcs neutralisants sur Metazoan

\begin{frame}[t]
  \frametitle{Use of Neutralizing Edges}

\makenoprio

\begin{tikzpicture}
  \path[use as bounding box] (-2,0) rectangle (8,6.5);
  \exmetazoan

  \node (nea1) at (3.5,5) {};
  \node[dotne] (nea2) at (0.6,5.9) {};
  \draw[linene] (nea1) to[out=120, in=20] (nea2);

  \node (neb1) at (5.8,3.6) {};
  \node[dotne] (neb2) at (-0.5,2.5) {};
  \draw[linene] (neb1) to[out=-70, in=250, min distance=160] (neb2);
\end{tikzpicture}

\end{frame}

\subsection{Synchronous Actions}
% Actions plurielles

\begin{frame}[c]
  \frametitle{Process Hitting with Synchronous Actions}

\begin{tikzpicture}
  \path[use as bounding box] (-5.2,-4) rectangle (5.2,3.5);
  \planPHstandard
  \planPHp
  \planPHan
  \planPHmult
  \planPHcanonique[stillhidden]
\end{tikzpicture}

\end{frame}



\begin{frame}[c]
  \frametitle{Addition of Synchronous Actions}

\begin{columns}
\begin{column}{.4\textwidth}

\begin{tikzpicture}
  %\path[use as bounding box] (-2,0) rectangle (8,6.5);
  %\TSort{(0,0)}{b}{2}{l}
  \TSort{(0,2)}{y}{2}{l}
  \TSort{(3,0)}{z}{2}{r}
  \TSort{(3,3)}{x}{2}{r}
  
  \TActionPlur{y_1}{x_1.west}{x_0.north west}{}{1.5,2.5}{right}
  \TActionPlur{}{z_0.west}{z_1.south west}{}{1.5,2.5}{left}
  \TActionPlur{}{x_0.east}{x_1.south east}{}{4,2.5}{right}

  \TState{1}{x_1, y_1, z_0}
  \TState{2}{x_0, y_1, z_1}
  \TState{3}{x_1, y_1, z_1}
\end{tikzpicture}

\end{column}
\begin{column}{.55\textwidth}
\begin{center}

\begin{itemize}
  \item Synchronizations between actions:
  \begin{itemize}
    \item[--] All catalysts must be present
    \item[--] Reactants are consumed all together
    \item[--] Simultaneous creation of the products
  \end{itemize}
  \item Representation of biochemical equations:\\
    \centering $X \xrightarrow{Y} Z$\\
    \raggedright under the form:\\
    \centering $h_2 = \PHfrappemults{x_1, y_1, z_0}{x_0, z_1}$\\
\end{itemize}

% \vspace*{.5cm}

% \ex{%
% $h_2 = \PHfrappemults{x_1, y_1, z_0}{x_0, z_1}$\\
% $h_1 = \PHfrappemults{c_0}{c_1}$%
% }

\vspace*{.5cm}
All processes of $A$\\
must be present to play $\PHfrappemult{A}{B}$

\medskip
After the play of $\PHfrappemult{A}{B}$,\\
all processes of $B$ are present
% 
% est jouable dans $s$ si et seulement si :
% 
% $A \subset s$
% 
% \bigskip
% Après jeu : $s \play (\PHfrappemult{A}{B}) = s \Cap B$

\end{center}
\end{column}
\end{columns}

\end{frame}

% Actions plurielles sur Metazoan

\begin{frame}[t]
  \frametitle{Use of Synchronous Actions}

\makenoprio

\begin{tikzpicture}[apdotsimple/.style={apdot}]
  \path[use as bounding box] (-3,1) rectangle (6,6.5);
  \TSort{(0,4)}{c}{2}{l}
  \TSort{(1,0)}{f}{2}{l}
  \TSort{(5,4)}{a}{2}{r}

  \TActionPlur{f_1, c_0}{a_0.west}{a_1.south west}{}{2.5,2.5}{left}
%   \THit{fc_2}{\prio}{a_0}{.west}{a_1}
%   \path[bounce, bend left=50]
%     \TBounce{a_0}{\prio}{a_1}{.south west};
  
  \THit{c_1}{\prio}{a_1}{.west}{a_0}
  \path[bounce, bend right=50]
    \TBounce{a_1}{\prio}{a_0}{.north west};
  
  \THit{f_1}{bend left=30, in=90}{c_0}{.west}{c_1}
  \path[bounce, bend left=50]
    \TBounce{c_0}{}{c_1}{.south west};
  
  \TActionPlur{}{c_1.west}{c_0.north west}{}{-1,6}{right}
%   \THit{c_1}{selfhit}{c_1}{.west}{c_0}
%   \path[bounce, bend right=50]
%     \TBounce{c_1}{}{c_0}{.north west};
\end{tikzpicture}

\begin{flushright}
  \f Same dynamics than classes of priorities,\\
  except for of the missing cooperative sort
\end{flushright}

\end{frame}


\section{Canonical Process Hitting and Analysis}
\subsection{Canonical Process Hitting}
% Ajout des actions prioritaires pour avoir une équivalence avec les ADN

\begin{frame}[c]
  \frametitle{Canonical Process Hitting}

\begin{tikzpicture}
  \path[use as bounding box] (-5.2,-4) rectangle (5.2,3.5);
  \planPHstandard
  \planPHp
  \planPHan
  \planPHmult
  \planPHcanonique
\end{tikzpicture}

\end{frame}



\begin{frame}[t]
  \frametitle{Temporal Shift in Cooperative Sorts}
  \framesubtitle{\tcite{\cfpmrcsbio}}

\begin{center}\scalebox{\scaleex}{
\begin{tikzpicture}
  \path[use as bounding box] (-0.5,-0.5) rectangle (6.5,4.5);
  %\path[use as bounding box] (-1,-0.5) rectangle (7.5,5);

  \exphcoopprio{unprio}{}

  \node<5->[process,very thick] at (z_1.center) {?};

  \only<2>{
    \THit{a_1}{selfhit,hlb}{a_1}{.west}{a_0}
    \path[bounce,bend right,hlb] \TBounce{a_1}{}{a_0}{.north west} ;
  }
  \only<3>{
    \THit{b_0.south west}{bend left=90,hlb}{a_0}{.west}{a_1}
    \path[bounce,bend left,hlb] \TBounce{a_0}{}{a_1}{.south west} ;
  }
  \only<4>{
    \THit{b_1}{selfhit,hlb}{b_1}{.west}{b_0}
    \THit{a_0}{bend right=50,hlb}{b_0}{.west}{b_1}
    \path[bounce,bend right,hlb] \TBounce{b_1}{}{b_0}{.north west} ;
    \path[bounce,bend left,hlb] \TBounce{b_0}{}{b_1}{.south west} ;
  }

  \TState{5-6}{a_0, b_0, ab_0, z_0}
  \only<6>{
  \THit{b_0.south west}{hl,bend left=90}{a_0}{.west}{a_1}
  \path[bounce,bend left,hl] \TBounce{a_0}{}{a_1}{.south west} ;
  }
  \TState{7}{a_1, b_0, ab_0, z_0}
  \only<7>{
  \THit{a_1}{hl}{ab_0}{.west}{ab_2}
  \path[bounce,bend left,hl] \TBounce{ab_0}{}{ab_2}{.240} ;
  }
  \TState{8}{a_1, b_0, ab_2, z_0}
  \only<8>{
  \THit{a_1}{selfhit,hl}{a_1}{.west}{a_0}
  \path[bounce,bend right,hl] \TBounce{a_1}{}{a_0}{.north west} ;
  }
  \TState{9}{a_0, b_0, ab_2, z_0}
  \only<9>{
  \THit{a_0}{bend right=50,hl}{b_0}{.west}{b_1}
  \path[bounce,bend left,hl] \TBounce{b_0}{}{b_1}{.south west} ;
  }
  \TState{10}{a_0, b_1, ab_2, z_0}
  \only<10>{
  \THit{b_1}{hl}{ab_2}{.200}{ab_3}
  \path[bounce,bend left,hl] \TBounce{ab_2}{}{ab_3}{.south} ;
  }
  \TState{11}{a_0, b_1, ab_3, z_0}
  \only<11>{
  \THit{ab_3}{hl}{z_0}{.west}{z_1}
  \path[bounce,bend left,hl] \TBounce{z_0}{}{z_1}{.south} ;
  }
  \TState{12-}{a_0, b_1, ab_3, z_1}
\end{tikzpicture}
}\end{center}

\medskip

\tval{Drawback}: the cooperative sorts are too ``loose'' (temporal shift)

\medskip

$ \uncover<5->{\PHstate{a_0, b_0, ab_{00}, z_0}}
  \uncover<7->{\rightarrow\PHstate{a_1, b_0, ab_{00}, z_0}}
  \uncover<8->{\rightarrow\PHstate{a_1, b_0, ab_{10}, z_0}}
  \uncover<9->{\rightarrow\PHstate{a_0, b_0, ab_{10}, z_0}}$
\\ \qquad
$ \uncover<10->{\rightarrow\PHstate{a_0, b_1, ab_{10}, z_0}}
  \uncover<11->{\rightarrow\PHstate{a_0, b_1, \redex{ab_{11}}, z_0}}
  \uncover<12->{\rightarrow\PHstate{a_0, b_1, \redex{ab_{11}}, \redex{z_1}}}$

\medskip

\uncover<12->{
\begin{tabular}{ll}
  Expected behavior: & \ex{$a_1 \wedge b_1$~\tval{simultaneously}} \quad \cad ``in the same state''
\\
  Obtained behavior: & \ex{$\mathbf{P}(a_1) \wedge \mathbf{P}(b_1)$} \quad with $\mathbf{P}$ = ``previously''
\end{tabular}
}
\end{frame}


\begin{frame}[t]
  \frametitle{Canonical Process Hitting}
  \framesubtitle{\tcite{\cfpmrcsbio}}

\begin{center}\scalebox{\scaleex}{
\begin{tikzpicture}
  \path[use as bounding box] (-0.5,-0.5) rectangle (6.5,4.5);

  \draw[draw=none, fill=blue!15] (2,2.1) ellipse (15pt and 2cm);
  \draw[draw=none, fill=red!15] (-1,2) ellipse (22pt and 10pt);
  \exphcoopprio{prio}{}
  \node[labelprio1] at (2,3.7) {$1$};
  
  \node[labelprio2] at (-.9,4.7) {$2$};  % a 1 -> a 1 0
  \node[labelprio2] at (-1.4,2) {$2$};  % b 0 -> a 0 1
%  \node[labelprio2] at (.7,2) {$2$};  % b 1 -> b 1 0
  \node[labelprio2] at (5.8,2.6) {$2$};  % ab 11 -> z 0 1

  \node[process,very thick] at (z_1.center) {?};

  \TState{3}{a_0, b_0, ab_0, z_0}
  \only<3>{
  \THit{b_0.south west}{hl,bend left=90}{a_0}{.west}{a_1}
  \path[bounce,bend left,hl] \TBounce{a_0}{}{a_1}{.south west} ;
  }
  \TState{4}{a_1, b_0, ab_0, z_0}
  \only<4>{
  \THit{a_1}{prio,hl}{ab_0}{.west}{ab_2}
  \path[bounce,bend left,hl] \TBounce{ab_0}{}{ab_2}{.240} ;
  }
  \TState{5}{a_1, b_0, ab_2, z_0}
  \only<5>{
  \THit{a_1}{selfhit,hl}{a_1}{.west}{a_0}
  \path[bounce,bend right,hl] \TBounce{a_1}{}{a_0}{.north west} ;
  }
  \TState{6}{a_0, b_0, ab_2, z_0}
  \only<6>{
  \THit{a_0}{prio,hl}{ab_2}{.160}{ab_0}
  \path[bounce,bend right,hl] \TBounce{ab_2}{}{ab_0}{.north west} ;
  }
  \TState{7}{a_0, b_0, ab_0, z_0}
  \only<7>{
  \THit{a_0}{bend right=50,hl}{b_0}{.west}{b_1}
  \path[bounce,bend left,hl] \TBounce{b_0}{}{b_1}{.south west} ;
  }
  \TState{8}{a_0, b_1, ab_0, z_0}
  \only<8>{
  \THit{b_1}{prio,hl}{ab_0}{.210}{ab_1}
  \path[bounce,bend left] \TBounce{ab_0}{hl}{ab_1}{.240} ;
  }
  \TState{9-}{a_0, b_1, ab_1, z_0}
\end{tikzpicture}
}\end{center}

%\newcommand{\intextpriolabel}[2]{\raisebox{-3pt}{\scalebox{#1}{\tikz \node[labelprio#2] {$#2$};}}}

\begin{itemize}
  \item Primary actions (updating cooperative sorts)
    \f \raisebox{-2pt}{\scalebox{\scaleex}{\tikz \node[labelprio1] {$1$};}}
    \begin{itemize}
      \item[] non-biological / non-controllable actions
    \end{itemize}
  \item Secondary actions (all the other ones)
    \f \raisebox{-2pt}{\scalebox{\scaleex}{\tikz \node[labelprio2] {$2$};}}
    \begin{itemize}
      \item[] biological / controllable actions / with delays
    \end{itemize}
\end{itemize}
\pause
$\Rightarrow$ Whenever a secondary action is played, all cooperative sorts are already updated

%\medskip
%Now, $z_1$ cannot be reached from $\PHstate{a_0, b_0, ab_{00}, z_0}$

\medskip
$ \uncover<3->{\PHstate{a_0, b_0, ab_{00}, z_0}}
  \uncover<4->{\rightarrow\PHstate{a_1, b_0, ab_{00}, z_0}}
  \uncover<5->{\rightarrow\PHstate{a_1, b_0, ab_{10}, z_0}}
  \uncover<6->{\rightarrow\PHstate{a_0, b_0, ab_{10}, z_0}}$
\\ \qquad
$ \uncover<7->{\rightarrow\PHstate{a_0, b_0, \ex{ab_{00}}, z_0}}
  \uncover<8->{\rightarrow\PHstate{a_0, b_1, \ex{ab_{00}}, z_0}}
  \uncover<9->{\rightarrow\PHstate{a_0, b_1, \ex{ab_{01}}, z_0}}$
\end{frame}



\begin{frame}[t]
  \frametitle{Canonical Process Hitting with Synchronous Actions}
%  \framesubtitle{\tcite{\cfpmrcsbio}}

\begin{center}\scalebox{\scaleex}{
  \begin{tikzpicture}[apdotsimple/.style={apdot}]
    \TSort{(0,0)}{a}{2}{l}
    \TSort{(0,3)}{b}{2}{l}
    \TSort{(4,2)}{z}{2}{r}
    
    \TActionPlur{a_1, b_1}{z_0.west}{z_1.south west}{}{2,2}{left}
    
%    \TAction{a_1}{a_1.west}{a_0.north west}{selfhit}{right}
    \TActionPlur{}{a_1.east}{a_0.north east}{}{1.5,0.75}{left}
%    \TAction{b_1}{b_1.west}{b_0.north west}{selfhit}{right}
    \TActionPlur{}{b_1.east}{b_0.north east}{}{1.5,4.25}{left}
    \TAction{a_0.south west}{b_0.west}{b_1.south west}{bend left=90}{left}
    \TAction{b_0}{a_0.west}{a_1.south west}{bend right=50}{left}
    
  \TState{1-}{a_0, b_0, z_0}
  \end{tikzpicture}
}\end{center}

\begin{itemize}
  \item Equivalent dynamics
  \item Sub-class of synchronous automata networks
  \item No priorities (no ill-formed model)
  \item No interfering updates and less intertwining
\end{itemize}

% \pause
% $\Rightarrow$ Whenever a secondary action is played, all cooperative sorts are already updated
% 
% %\medskip
% %Now, $z_1$ cannot be reached from $\PHstate{a_0, b_0, ab_{00}, z_0}$
% 
% \medskip
% $ \uncover<3->{\PHstate{a_0, b_0, ab_{00}, z_0}}
%   \uncover<4->{\rightarrow\PHstate{a_1, b_0, ab_{00}, z_0}}
%   \uncover<5->{\rightarrow\PHstate{a_1, b_0, ab_{10}, z_0}}
%   \uncover<6->{\rightarrow\PHstate{a_0, b_0, ab_{10}, z_0}}$
% \\ \qquad
% $ \uncover<7->{\rightarrow\PHstate{a_0, b_0, \ex{ab_{00}}, z_0}}
%   \uncover<8->{\rightarrow\PHstate{a_0, b_1, \ex{ab_{00}}, z_0}}
%   \uncover<9->{\rightarrow\PHstate{a_0, b_1, \ex{ab_{01}}, z_0}}$
\end{frame}

\subsection{Static Analysis}
% Analyse statique avec priorités

\begin{frame}[c]
  \frametitle{Static Analysis of Canonical Process Hitting}
  \framesubtitle{\tcite{\cfpmrcsbio}}
%  \framesubtitle{\tcite{\cpmrmscs}}

Adding priorities restricts the possible dynamics (preemptions)

\smallskip
\f Invalidates the previous under-approximation

\begin{center}
\scalebox{0.6}{
\begin{tikzpicture}
  \path[use as bounding box] (-5,-3.5) rectangle (5,3.5);
  \definecolor{r2}{RGB}{238,10,38}

  \draw[shading=2, inner color=r2, outer color=white, rounded corners, draw=none] (-6,3.5) rectangle (6,-3.5);
  \draw[thick,fill=white] (-2.8,2) rectangle (2.8,-2);
  \draw<4->[thick,fill=lightyellow] (-2.8,2) rectangle (2.8,-2);
  \draw<1>[thick, shading=1, top color=darkgreen, bottom color=green,opacity=1] (-1.5,.7) rectangle (1.5,-.7);;
  \draw<2->[thick, shading=1, top color=darkgreen, bottom color=green,opacity=0.3] (-1.5,.7) rectangle (1.5,-.7);;
%  \draw<1>[color=darkcyan, ultra thick] (0,0) ellipse (2 and 1.5);
  \draw<1>[color=darkcyan, ultra thick] (1,1.3) arc [start angle=60, end angle=420, x radius=2, y radius=1.5] -- cycle;
  \draw<2->[color=darkcyan, ultra thick] (1,1.3) arc [start angle=60, end angle=300, x radius=2, y radius=1.5] -- cycle;
  \draw<3->[thick, shading=1, top color=darkgreen, bottom color=green] (-1.5,.7) rectangle (.8,-.7);;
\end{tikzpicture}
}
\end{center}

\uncover<5->{
Similar complexity for a more expressive formalism

\begin{fleches}
  \item Still efficient for big models
  \item Finer under-approximation
\end{fleches}
}
\end{frame}


\begin{frame}[c]
  \frametitle{Static Analysis of Canonical Process Hitting}
  \framesubtitle{\tcite{\cfpmrcsbio}}

\uncover<8->{
\tval{Sufficient condition:}

\begin{itemize}
  \item no cycle
  \item each objective has a solution
  \item \only<-11>{cooperations are coherent}\only<12->{\sout{cooperations are coherent}}
\end{itemize}
\vspace{1cm}
\hspace{2cm}\uncover<12->{\textcolor{darkyellow}{\textbf{Non-conclusive}}}
\vspace{-3cm}
}

\begin{center}\scalebox{\scaleex}{
\begin{tikzpicture}[aS]
  \path[draw=none,use as bounding box] (-.5,-2.2) rectangle (12,2.2);
  \node[Aproc] (z1) {$z_1$};
  \uncover<2->{ \node[Aobj,right of=z1] (z01) {$\PHobj{z_0}{z_1}$}; }
  \uncover<3->{ \node[Asol,right of=z01] (z01s) {}; }
  
  \uncover<4->{ \node[Aproc,right of=z01s] (ab11) {$ab_{11}$}; }
  \uncover<5->{ \node[Asol,right of=ab11] (ab11s) {}; }
  
  \uncover<6->{
    \node[Aproc,above right of=ab11s] (a1) {$a_1$};
    \node[Aproc,below right of=ab11s] (b1) {$b_1$};
  }
  
  \uncover<7->{
    \node[Aobj,above right of=a1] (a11) {$\PHobj{a_1}{a_1}$};
    \node[Asol,right of=a11] (a11s) {};
    \node[Aobj,right of=a1] (a01) {$\PHobj{a_0}{a_1}$};
    \node[Asol,right of=a01] (a01s) {};
    \node[Aproc,right of=a01s] (b0) {$b_0$};
    \node[Aobj,right of=b0] (b00) {$\PHobj{b_0}{b_0}$};
    \node[Asol,right of=b00] (b00s) {};
    \node[Aobj,above right of=b0] (b10) {$\PHobj{b_1}{b_0}$};
    \node[Asol,right of=b10] (b10s) {};
    
    \node[Aobj,below right of=b1] (b11) {$\PHobj{b_1}{b_1}$};
    \node[Asol,right of=b11] (b11s) {};
    \node[Aobj,right of=b1] (b01) {$\PHobj{b_0}{b_1}$};
    \node[Asol,right of=b01] (b01s) {};
    \node[Aproc,right of=b01s] (a0) {$a_0$};
    \node[Aobj,right of=a0] (a00) {$\PHobj{a_0}{a_0}$};
    \node[Asol,right of=a00] (a00s) {};
    \node[Aobj,below right of=a0] (a10) {$\PHobj{a_1}{a_0}$};
    \node[Asol,right of=a10] (a10s) {};
  }
  
  \path (z1) edge (z01);
  \path<2-> (z01) edge (z01s);
  \path<3-> (z01s) edge (ab11);
  \path<4-> (ab11) edge[aSPrio] (ab11s);
  \path<5-> (ab11s) edge (a1) edge (b1);
  \path<6-> (a1) edge (a01) edge (a11) (b1) edge (b01) edge (b11);
  
  \path<7->
  (a01) edge (a01s)
  (a01s) edge (b0)
  (a11) edge (a11s)
  (a0) edge (a10) edge (a00)
  (a10) edge (a10s)
  (a00) edge (a00s)
  
  (b0) edge (b10) edge (b00)
  (b10) edge (b10s)
  (b00) edge (b00s)
  (b01) edge (b01s)
  (b01s) edge (a0)
  (b11) edge (b11s)
  ;
  
  % Arc non cohérent
  \node<9-11>[Aproc,Aex,at=(ab11)] {$ab_{11}$};
  \node<9-11>[Asol,Aexsol,right of=ab11] (ab11s) {};
  \path<9-11> (ab11) edge[aSPrio,Aexedge] (ab11s);
  \node<12>[Aproc,Ahl,at=(ab11)] {$ab_{11}$};
  \node<12>[Asol,Ahlsol,right of=ab11] (ab11s) {};
  \path<12> (ab11) edge[aSPrio,Ahledge] (ab11s);
  
  \node<10>[Aproc,Aex,at=(a1)] {$a_1$};
  \node<10>[Aproc,Aex,at=(b1)] {$b_1$};
  \node<11->[Aproc,Ahl,at=(a1)] {$a_1$};
  \node<11->[Aproc,Ahl,at=(a0)] {$a_0$};
\end{tikzpicture}
}\end{center}

\scalebox{\scaleminiex}{
\begin{tikzpicture}
  \path[use as bounding box] (-0.5,-0.5) rectangle (8.5,3.5);
  \tikzstyle{current process}=[process,fill=gray]
  \exphcoopprio{prio}{}
  \node[process,very thick] at (z_1.center) {?};
  \TState{1-}{a_0, b_0, ab_0, z_0}
\end{tikzpicture}}
\hfill
\scalebox{\scaleex}{
\scalebox{\scaleex}{
\begin{tikzpicture}[aS]
  \path[use as bounding box] (0,0) rectangle (5.8,4);
  \glclegend{prio}{$z_1$}{$\PHobj{z_0}{z_1}$}
\end{tikzpicture}
}}

\end{frame}



\begin{frame}[c]
  \frametitle{Implementation of the Static Analysis Into \Pint}

Complexity:

\begin{itemize}
  \item Computation of the local causality graph:
  \begin{itemize}
    \item Polynomial in the number of sorts
    \item Exponential in the number of processes of each sort
  \end{itemize}
  \item Analysis of the graph (sufficient condition):
  \begin{itemize}
    \item Polynomial in the size of the graph
  \end{itemize}
\end{itemize}

\pause
\medskip
Makes the study of large networks tractable:

\bigskip
\small
\begin{tabular}{r||c|c|c|c||c|c|c|}
%\hline
\tval{Modèle} & Sortes & Processus & Actions & États & libddd$^1$ & GINsim$^2$ & \cellcolor{couleurtitre} \Pint \\\hline
\tval{\ex{egfr20}} & 35 & 196 & 670 & $2^{64}$ & & $<$1s & \cellcolor{couleurtitre} \tval{0.02s} \\\hline
\tval{\ex{tcrsig40}} & 54 & 156 & 301 & $2^{73}$ & & $\infty$ & \cellcolor{couleurtitre} \tval{0.02s} \\\hline
\tval{\ex{tcrsig94}} & 133 & 448 & 1124 & $2^{194}$ & [13min -- $\infty$] & & \cellcolor{couleurtitre} \tval{0.03s} \\\hline
\tval{\ex{egfr104}} & 193 & 748 & 2356 & $2^{320}$ & & & \cellcolor{couleurtitre} \tval{0.16s}\\\hline
\end{tabular}

\bigskip
\footnotesize
\quad$^1$ LIP6/Move
\tcite{Couvreur \textit{et al.}, \textit{Lecture Notes in Computer Science}, 2002}\\
\quad$^2$ TAGC/IGC
%\tcite{Gonzalez \textit{et al.}, \textit{Biosystems}, 2006}
\tcite{Chaouiya, Naldi, Thieffry, \textit{Methods in Molecular Biology}, 2012}

%S = Sorts \quad CS = Cooperative sorts \quad P = Processes \quad A = Actions

%\cmodels
%\medskip
%\todo{Citer les papiers d'origine}
% \citeegfra\\
% \citetcrsiga\\
% \citetcrsigb\\
% \citeegfrb\\
\cmodels

\end{frame}

\subsection{Translations}

\begin{frame}[c]
  \frametitle{Formal Translation Into Canonical Form}

\makenoprio

\vspace*{.5cm}
\scalebox{.9}{
\begin{tikzpicture}
  \path[use as bounding box] (-5.75,0) rectangle (5.75,5.5);
  \TSort{(-5,4)}{c}{2}{l}
  \TSort{(0,1)}{f}{2}{l}
  \TSort{(5,4)}{a}{2}{r}

  \TSetTick{fc}{0}{00}
  \TSetTick{fc}{1}{01}
  \TSetTick{fc}{2}{10}
  \TSetTick{fc}{3}{11}
  \TSort{(3,0)}{fc}{4}{r}
  
  \THit{fc_2}{}{a_0}{.south west}{a_1}
  \path[bounce, bend left=60]
    \TBounce{a_0}{}{a_1}{.south west};
  
  \THit{c_1.north east}{}{a_1}{.west}{a_0}
  \path[bounce, bend right=50]
    \TBounce{a_1}{}{a_0}{.north west};
  
  \path (0.8, 1.5) edge[\prio,coopupdate] (2.2, 1.5);
  \path (-4.3, 4.5) edge[\prio,coopupdate] (2.2, 2.5);
  
  \only<1>{
    \THit{c_1.north}{selfhit}{c_1}{.west}{c_0}
    \path[bounce, bend right=50]
      \TBounce{c_1}{}{c_0}{.north west};
  }
  
  \only<2>{
    \THit{c_1.north}{selfhit,hlr}{c_1}{.west}{c_0}
    \path[bounce, bend right=50]
      \TBounce{c_1}{hlr}{c_0}{.north west};
  }
  
  \only<-2>{
    \node[labelprio3] at (-4.4,6) {$3$};
  }
  
  \only<3>{
    \THit{a_0.west}{hlv}{c_1}{.east}{c_0}
    \path[bounce, bend left=50]
      \TBounce{c_1}{hlv}{c_0}{.north east};
    \node[labelprio2] at (0.5,4.7) {$2$};
  }
  
  \only<4->{
    \THit{a_0.west}{}{c_1}{.east}{c_0}
    \path[bounce, bend left=50]
      \TBounce{c_1}{}{c_0}{.north east};
    \node[labelprio2] at (0.5,4.7) {$2$};
  }
  
  \only<-4>{
    \node[labelprio3] at (-3,2.3) {$3$};
  }
  
  \only<-3>{
    \THit{f_1}{bend left=30, in=90}{c_0}{.west}{c_1}
    \path[bounce, bend left=50]
      \TBounce{c_0}{}{c_1}{.south west};
  }
  
  \only<4>{
    \THit{f_1}{bend left=30, in=90, hlr}{c_0}{.west}{c_1}
    \path[bounce, bend left=50]
      \TBounce{c_0}{hlr}{c_1}{.south west};
  }
  
  \only<5->{
    \TSetTick{fa}{0}{00}
    \TSetTick{fa}{1}{01}
    \TSetTick{fa}{2}{10}
    \TSetTick{fa}{3}{11}
    \TSort{(-3,0)}{fa}{4}{l}
    
    \path (-0.8, 1.5) edge[\prio,coopupdate] (-2.2, 1.5);
    \path (4.3, 4.5) edge[\prio,coopupdate] (-2.2, 2.5);
    
    \only<5>{
      \THit{fa_3}{hlv}{c_0}{.east}{c_1}
      \path[bounce, bend right=50]
        \TBounce{c_0}{hlv}{c_1}{.south east};
    }
    
    \only<6->{
      \THit{fa_3}{}{c_0}{.east}{c_1}
      \path[bounce, bend right=50]
        \TBounce{c_0}{}{c_1}{.south east};
    }
  
    \node[labelprio1] at (-1.5,3) {$1$};
    \node[labelprio1] at (-1.5,1.8) {$1$};
    \node[labelprio2] at (-4,3.85) {$2$};
  }
  
  \node[labelprio1] at (1.5,3) {$1$};
  \node[labelprio1] at (1.5,1.8) {$1$};

  \node[labelprio2] at (0,5.4) {$2$};
  \node[labelprio2] at (4.4,3) {$2$};
\end{tikzpicture}
}

\pause[6]
\vspace{.7cm}
\begin{center}
  \f Same dynamics (with supplemental cooperative sorts)
  
  \medskip
  \f The canonical form can be computed for all Process Hitting extensions,\\
    with classes of priorities, neutralizing edges or synchronous actions
\end{center}

\end{frame}

% Traductions et équivalences

\begin{frame}[c]
  \frametitle{Equivalence Between Process Hitting Extensions}

% \begin{center}
% \includegraphics[height=.5\textheight]{figs/PH1.png}
% \end{center}

\setbeamercovered{transparent}

\begin{center}
\scalebox{.8}{
\begin{tikzpicture}
  \path[use as bounding box] (-5.2,-4) rectangle (5.2,3.5);
  \planPHstandard
  \planPHp
  \planPHan
  \planPHmult
  \planPHcanonique
  
  \uncover<2->{
    \path[draw, very thick, bend right=10] (-0.2,2.2) edge[->] (-.2,1.6);
    \path[draw, very thick, bend left=10] (0.2,2.2) edge[<-] (.2,1.6);
  }
  
  \uncover<3->{
    \path[draw, very thick, bend right=10] (-2,1.8) edge[->] (-3,.5);
    \path[draw, very thick, bend right=10] (-2,0) edge[->] (-1,1.1);
  }
  
  \uncover<4->{
    \path[draw, very thick, bend left=10] (2,1.8) edge[<-] (3,.5);
    \path[draw, very thick, bend left=10] (2,0) edge[<-] (1,1.1);
  }
  
  \uncover<5->{
    \planPHstandardligne
    \path[draw, very thick, bend left=10] (0,-1.9) edge[->] (0,-.8);
  }
\end{tikzpicture}
}
\end{center}

All developed enrichments have the same expressivity
\begin{itemize}
  \item Expressive power improved
  \item Can always be translated to the canonical form
  \item But sometimes at the cost of an exponential translation
\end{itemize}

\end{frame}



\setbeamercovered{transparent}

\begin{frame}[c]
  \frametitle{Translation From and To Other Discrete Models}

% \begin{center}
% \hspace*{-1cm}\includegraphics[height=.6\textheight]{figs/PH2.png}
% \end{center}

\begin{center}
\scalebox{.7}{
\begin{tikzpicture}
  \path[use as bounding box] (-5.2,-4) rectangle (5.2,3.5);
  \planPHstandard
  \planPHp
  \planPHan
  \planPHmult
  \planPHcanonique
  \planPHstandardligne
  
  \uncover<2->{
    \draw[thick, draw=black, fill=gray!10] (-4.5,2.5) ellipse (1.4 and .6)
      node[text width=4cm, align=center] {\small Modèle de Thomas\\Réseaux discrets};
    \path[draw, very thick, bend left=5] (-3.3,2.5) edge[->] (-1.5,1.4);
    \path[draw, very thick, bend right=5] (-3.4,2.3) edge[<-] (-1.6,1.2);
  }
  
  \uncover<3->{
    \draw[thick, draw=black, fill=gray!10] (6,2) ellipse (1.2 and .5)
      node[text width=4cm, align=center] {\small Automates\\synchronisés};
    \path[draw, very thick, bend left=5] (6,1.6) edge[->] (4.7,0.4);
    \path[draw, very thick, bend right=5] (5.7,1.7) edge[<-] (4.4,0.5);
  }
  
  \uncover<4->{
    \draw[thick, draw=black, fill=gray!10] (4.2,3.2) ellipse (1.7 and .5)
      node[text width=4cm, align=center] {\small Sémantique booléenne\\de Biocham};
    \path[draw, very thick, bend right=5] (4.2,2.8) edge[->] (3.7,0.6);
  }
  
  \uncover<5->{
    \draw[thick, draw=black, fill=gray!10] (6,-1.7) ellipse (1.7 and .5)
      node[text width=4cm, align=center] {\small Réseaux de Petri bornés\\avec arcs inhibiteurs};
    \path[draw, very thick, bend right=5] (5.7,-1.4) edge[<-] (4.2,-.6);
  }
\end{tikzpicture}
}
\end{center}

\begin{itemize}
  \item Equivalence with discrete networks / Thomas modeling
  \item Equivalence with synchronous automata networks
  \item Translation towards (bounded) Petri nets with inhibitor arcs
  \item Translation from the Boolean semantics of Biocham
\end{itemize}

\end{frame}

\setbeamercovered{invisible}

% Implémentation

\begin{frame}[c]
  \frametitle{Translation to Thomas Modeling}
  \framesubtitle{\tcite{\cfpimrcmsb}}

\begin{itemize}
  \item Two successive inferences: 1) interaction graph; 2) parameters
  \item Exhaustive analysis of the local dynamics for each regulator
  \item enumeration of all parametrizations compatible with the dynamics
\end{itemize}

\bigskip
\tval{Complexity}:\\
\quad Linear in the number of genes,\\
\quad Exponential in the number of regulators of one component

\pause
\bigskip
\small
\begin{tabular}{r||c|c|c||c|c||c|c|}
\multicolumn{4}{c||}{Models} & \multicolumn{2}{c||}{Inference the IG} & \multicolumn{2}{c|}{Inference of parameters}\\
\hline
\tval{Name} & Sorts & Processes & Actions & Duration & Edges & Durations & Parameters\\
\hline
  \tval{\ex{egfr20}} & 42 & 152 & 399 & \tval{1s} & 51 & \tval{1s} & 192\\
\hline
  \tval{\ex{tcrsig40}} & 54 & 156 & 305 & \tval{1s} & 55 & \tval{1s} & 143\\
\hline
  \tval{\ex{tcrsig94}} & 133 & 448 & 1082 & \tval{100s} & 197 & \tval{1s} & 578\\
\hline
  \tval{\ex{egfr104}} & 193 & 744 & 2304 & \tval{200s} & 280 & \tval{3s} & 27'496\\
\hline
\end{tabular}

%S = Sortes \quad CS = Sortes coopératives \quad P = Processus \quad A = Actions

\footnotesize
\cmodels
\end{frame}


\section{Conclusion}
% Conclusion

\begin{frame}[c]
  \frametitle{General Conclusion}

Standard Process Hitting allows to represent biological regulatory networks
in an \tval{atomistic} fashion:
\begin{itemize}
  \item Existing efficient static analysis
  \item But temporal shift issues
  \item Limited modeling power
\end{itemize}

\medskip
\tval{Extensions of the Process Hitting} to improve the expressivity:
\begin{itemize}
  \item Rectification of the temporal shift \f Strictly higher expressivity
  \item Allows to abstract temporal parameters
  \item New links to other formalisms (Thomas, PN, etc.)
\end{itemize}

\medskip

\tval{Static analysis} of the Canonical Process Hitting:
\begin{itemize}
  \item Efficient analysis of reachability properties
  \item Applicable to the extensions at the cost of a translation
  \item New kind of property: simultaneous activation
\end{itemize}

% 
% 
% Process Hitting: an atomistic modeling with powerful static analysis
% 
% \medskip
% \begin{enumerate}[1.]
%   \item Stochastic parameters:
%     \begin{itemize}
%       \item To model systems with chronometric features
%       \item \tval{Continuous time}
%       \item But \tval{hard to analyze}
%     \end{itemize}
%   \item Classes of priorities:
%     \begin{itemize}
%       \item Allows to reproduce the same behaviors
%       \item Efficient \tval{static analysis}
%       \item But the translation to canonical form faces \tval{combinatorial explosion}
%     \end{itemize}
%   \item Neutralizing edges:
%     \begin{itemize}
%       \item Alternative to priorities
%       \item Closer to reality in some cases
%       \item \tval{Lighter translation} to canonical form
%     \end{itemize}
% \end{enumerate}
% 
% \vfill
% \Large
% \begin{flushright}
%   \tval{Thank you}\hspace{1cm}~
% \end{flushright}
% \vfill

\end{frame}

\setbeamercovered{invisible}



\begin{frame}[c]
  \frametitle{Outlooks}

New \tval{exploitation} possibilities:
\begin{itemize}
  \item Modeling and analysis of full databases
  \item Study of uncontrollable behaviors or punctual perturbations
  \item Research of interesting properties (attractors, oscillations, ...)
\end{itemize}

\medskip
Improvement of the \tval{static analysis}:
\begin{itemize}
  \item Refining in order to reduce the non-conclusiveness
  \item New methods using by-products such as the local causality graph
  \item New properties to check (temporal logic, counters, ...)
\end{itemize}

\medskip
Enrichment of the \tval{modeling power}:
\begin{itemize}
  \item Abstraction of temporal parameters: find properties to avoid Zeno behavior
  \item Dynamical classes of priorities
  \item Guarded actions or complex logic gates
  \item New model checking tools (Hoare logic, ...)
\end{itemize}

\end{frame}


% 
% \begin{frame}[c]
%   \frametitle{Collaborations}
% 
% Participation to the \tval{ANR blanc} project \tval{BioTempo} (March 2011 -- November 2014):
% \begin{center}
% “Language, time representations and hybrid models\\
% for the analysis of incomplete models in molecular biology”
% \end{center}
% Task 3: Introduce synchronization and continuous time in chronological models:\\
% programming language, multi-clocks and hybrid systems
% 
% \bigskip
% \bigskip
% 3 months PhD internship (March -- May 2012):\\
% \tval{National Institute of Informatics} (Tokyo, Japan)\\
% Invited in the team of \tval{Katsumi Inoue}
% \begin{center}
% “Automated Reasoning and Hypothesis\\
% Finding for Systems Biology”
% \end{center}
% Partnership organized with AtlanSTIC
% Financial participation of Centrale Initiatives 
% 


%\tval{Inoue Laboratory} (NII, Sokendai): Constraint Programming, Systems Biology

%\tval{MeForBio} (IRCCyN, ÉCN): Formal Methods for Bioinformatics

%\tval{AMIB} (LIX, Polytechnique): Algorithms and Models for Integrative Biology

% \bigskip\footnotesize
% \begin{center}
%   $\left.\text{\begin{tabular}{ccc}
%       \includegraphics[height=1.5cm]{figs/Olivier.jpg}
%     & \includegraphics[height=1.5cm]{figs/Morgan.jpg} \\
%       \tval{Olivier ROUX} & \tval{Morgan MAGNIN} \\
%       Professeur \& chef d'équipe & Maître de conférences
%   \end{tabular}}\right\}$ %\text{\tval{MeForBio}}$%}
%   \parbox{2cm}{\tval{MeForBio}\\IRCCyN\\(Nantes, France)}
% 
%   \vspace*{3em}
%   $\left.\text{\begin{tabular}{c}
%     \includegraphics[height=1.5cm]{figs/Loic.jpg} \\ \tval{Loïc PAULEVÉ} \\ Chargé de recherche CNRS
%   \end{tabular}}\right\}$%\text{\tval{AMIB}}$
%   \parbox{1.5cm}{\tval{Bioinfo/AMIB}\\LRI\\(Orsay, France)}
%   \hspace*{3em}
%   $\left.\text{\begin{tabular}{c}
%     \includegraphics[height=1.5cm]{figs/Inoue-sensei.jpg} \\ \tval{Katsumi INOUE} \\ Professeur \& chef d'équipe
%   \end{tabular}}\right\}$%\text{\tval{Inoue Laboratory}}$
%   \parbox{1.5cm}{\tval{Inoue Lab.}\\NII\\(Tokyo, Japon)}
% \end{center}

% \end{frame}

% 
% 
% \begin{frame}[c]
%   \frametitle{Personal Contributions}
% 
% \small
% \emphcolor{Book chapter:}
% \begin{itemize}
%   \item Paulevé, Chancellor, \tval{Folschette}, Magnin, Roux :\\
%     \tval{Analyzing Large Network Dynamics with Process Hitting},\\
%     \textit{Logical Modeling of Biological Systems}, août 2014
% \end{itemize}
% 
% \medskip
% \emphcolor{Conferences and workshops:}
% \begin{itemize}
%   \item \tval{Folschette}, Paulevé, Magnin, Roux :\\
%     \tval{Under-approximation of reachability in multivalued asynchronous networks},
%     \begin{tikzpicture}
%       \path[use as bounding box] (0,0) rectangle (0,.1);
%       \path[grosarca, ->] (2,0.08) edge (.3,0.08);
%       \path[grosarca, ->] (2,-3.45) edge (.3,-3.45);
%       \path[grosarca] (2,0.08) edge (2,-3.45);
%     \end{tikzpicture}\\
%     CS2Bio'13, \textit{Electronic Notes in Theoretical Computer Science}, \vol 299, 2013\\
%     \emphcolor{sélectionné pour un numéro spécial} de \textit{Theoretical Computer Science}
%  \item \tval{Folschette}, Paulevé, Inoue, Magnin, Roux :\\
%     \tval{Concretizing the process hitting into biological regulatory networks},
%     \begin{tikzpicture}
%       \path[use as bounding box] (0,0) rectangle (0,.1);
%       \path[grosarcb, ->] (2,0.08) edge (0.3,0.08);
%       \path[grosarcb, ->] (2,-0.9) edge (0.8,-0.9);
%       \path[grosarcb, ->] (2,-3.15) edge (1.1,-3.15);
%       \path[grosarcb] (2,0.08) edge (2,-3.15);
%     \end{tikzpicture}\\
%     CMSB'12, \textit{Lecture Notes in Computer Science}, 2012
%   \item \tval{Folschette}, Paulevé, Inoue, Magnin, Roux :\\
%     \tval{Abducing Biological Regulatory Networks from Process Hitting models},\\
%     \textit{ECML-PKDD'12 / LDSSB'12}, 2012
% \end{itemize}
% 
% \emphcolor{Current journal submissions:}
% \begin{itemize}
%   \item \tval{Folschette}, Paulevé, Magnin, Roux :\\
%     \tval{Sufficient Conditions for Reachability in Automata Networks with Priorities},\\
%     \emphcolor{soumis} à un numéro spécial de \textit{Theoretical Computer Science}
%   \item \tval{Folschette}, Paulevé, Inoue, Magnin, Roux :\\
%     \tval{Constructing Biological Regulatory Networks from Process Hitting models},\\
%     \emphcolor{en cours de révision} pour \textit{Theoretical Computer Science}
%   \item Paulevé, \tval{Folschette}, Magnin, Roux :\\
%     \tval{Analyses statiques de la dynamique des réseaux d'automates indéterministes},\\
%     \emphcolor{soumis} à un numéro spécial de \textit{Technique et Science Informatiques}
% \end{itemize}
% 
% \end{frame}
% 


% 
% \begin{frame}[c]
%   \frametitle{Contributions personnelles}
% 
% \small
% \emphcolor{Chapitre de livre} avec Paulevé, Chancellor, Magnin, Roux :
% \begin{itemize}
%   \item \tval{Analyzing Large Network Dynamics with Process Hitting},\\
%     \textit{Logical Modeling of Biological Systems},
% %    éditeurs : Luis Farinas del Cerro et Katsumi Inoue,
%     2014%, ISBN 978-1-84821-680-8.
% \end{itemize}
% 
% \medskip
% \emphcolor{Workshop} avec Paulevé, Magnin, Roux :
% \begin{itemize}
%   \item \tval{Under-approximation of reachability in multivalued asynchronous networks},\\
%     CS2Bio'13,
%     %in: Proceedings of the fourth International Workshop on Interactions between Computer Science and Biology,
%     %éditeurs : Emanuela Merelli et Angelo Troina,
%     \textit{Electronic Notes in Theoretical Computer Science}, \vol 299, 2013\\
%     \emphcolor{sélectionné pour un numéro spécial} de \textit{Theoretical Computer Science}
%     %33--51, Springer Berlin Heidelberg, juin 2013, DOI 10.1016/j.entcs.2013.11.004.
%     %\emphcolor{Selected for a special issue in the journal \textit{Theoretical Computer Science}.}
% \end{itemize}
% 
% \emphcolor{Conférence et workshop} avec Paulevé, Inoue, Magnin, Roux :
% \begin{itemize}
%   \item \tval{Concretizing the process hitting into biological regulatory networks},\\ %\newline{}
%     CMSB'12, \textit{Lecture Notes in Computer Science}, %éditeurs : David Gilbert et Monika Heiner, %\newline{}
%     2012
%     %in: \textit{Computational Methods in Systems Biology}, éditeurs : David Gilbert et Monika Heiner, %\newline{}
%     %166--186, Springer Berlin Heidelberg, octobre 2012, DOI 10.1007/978-3-642-33636-2\_11.
%   \item \tval{Abducing Biological Regulatory Networks from Process Hitting models},\\
%     \textit{ECML-PKDD'12 / LDSSB'12}, %éditeurs : Oliver Ray et Katsumi Inoue,
%     2012
%     %24--35, septembre 2012.
% \end{itemize}
% 
% \emphcolor{Soumissions de journaux en cours :}
% \begin{itemize}
%   \item \tval{Folschette}, Paulevé, Magnin, Roux :\\
%     \tval{Sufficient Conditions for Reachability in Automata Networks with Priorities},\\
%     %version étendue de “Under-approximation of reachability in multivalued asynchronous networks”,
%     \emphcolor{soumis} à un numéro spécial de  de \textit{Theoretical Computer Science}% en avril 2014.
%   \item \tval{Folschette}, Paulevé, Inoue, Magnin, Roux :\\
%     \tval{Constructing Biological Regulatory Networks from Process Hitting models},\\
%     %extended version of “Concretizing the process hitting into biological regulatory networks”,
%     \emphcolor{en cours de révision} pour \textit{Theoretical Computer Science}
%   \item Paulevé, \tval{Folschette}, Magnin, Roux :\\
%     \tval{Analyses statiques de la dynamique des réseaux d'automates indéterministes},\\
%     \emphcolor{soumis} à un numéro spécial de \textit{Technique et Science Informatiques}
%     %editors: N.~Fatès and S.~Sené,
%     %\emphcolor{soumis} en avril 2014.
% \end{itemize}
% 
% \end{frame}

\section[x]{Thank you for your attention}
% Merci !

\begin{frame}[c]

\vspace*{1cm}
\LARGE
\centering
\textbf{Thank you for your attention}

\end{frame}


\appendix
\newcounter{finalframe}
\setcounter{finalframe}{\value{framenumber}}

\section[x]{Bibliography}
% Bibliographie

\begin{frame}[c]
  \frametitle{Bibliography}

%\footnotesize
\small
\setlength{\parindent}{-1em}
\setlength{\parskip}{0.5em}

\tcitebullet Adrien Richard, Jean-Paul Comet, Gilles Bernot.
\emphcolor{R. Thomas' logical method}, 2008.
Invité à \textit{Tutorials on modelling methods and tools: Modelling a genetic switch and Metabolic Networks}, Spring School on Modelling Complex Biological Systems in the Context of Genomics.

\tcitebullet Stuart A. Kauffman.
\emphcolor{Metabolic stability and epigenesis in randomly constructed genetic nets}.
\textit{Journal of Theoretical Biology}, 22(3), pages 437--467, 1969.

\tcitebullet René Thomas.
\emphcolor{Boolean formalization of genetic control circuits}.
\textit{Journal of Theoretical Biology}, 42(3), pages 563--585, 1973.

\tcitebullet Élisabeth Remy, Paul Ruet and Denis Thieffry.
\emphcolor{Graphic requirements for multistability and attractive cycles in a boolean dynamical framework}.
\textit{Advances in Applied Mathematics}, 41(3), pages 335--350, Elsevier, 2008.

\tcitebullet Adrien Richard, Jean-Paul Comet.
\emphcolor{Necessary conditions for multistationarity in discrete dynamical systems}.
\textit{Discrete Applied Mathematics}, 155(18), pages 2403--2413, 2007.

\tcitebullet Gilles Bernot, Jean-Paul Comet, Adrien Richard and Janine Guespin.
\emphcolor{Application of formal methods to biological regulatory networks: extending Thomas' asynchronous logical approach with temporal logic}.
\textit{Journal of Theoretical Biology}, 229(3), pages 339--347, Elsevier, 2004.

\tcitebullet Sohei Ito, Naoko Izumi, Shigeki Hagihara and Naoki Yonezaki.
\emphcolor{Qualitative analysis of gene regulatory networks by satisfiability checking of Linear Temporal Logic}.
In 2010 IEEE International Conference on \textit{BioInformatics and BioEngineering}, pages 232--237, IEEE, 2010.

\end{frame}



\begin{frame}[c]
  \frametitle{Bibliographie}

\small
\setlength{\parindent}{-1em}
\setlength{\parskip}{0.5em}

\tcitebullet Loïc Paulevé, Morgan Magnin, Olivier Roux.
\emphcolor{Refining dynamics of gene regulatory networks in a stochastic $\pi$-calculus framework}.
In Corrado Priami, Ralph-Johan Back, Ion Petre, and Erik de Vink, editors: Transactions on Computational Systems Biology XIII,
\textit{Lecture Notes in Computer Science} 6575, pages 171--191, 2011.

\tcitebullet Loïc Paulevé, Morgan Magnin, Olivier Roux.
\emphcolor{Static analysis of biological regulatory networks dynamics using abstract interpretation}.
\textit{Mathematical Structures in Computer Science}, 2012.

\tcitebullet Paul François, Vincent Hakim, Eric D Siggia.
\emphcolor{Deriving structure from evolution : metazoan segmentation}.
\textit{Molecular Systems Biology}, 3(1), 2007.

\tcitebullet Özgür Sahin \textit{et al.}
\emphcolor{Modeling ERBB receptor-regulated G1/S transition to find novel targets for de novo trastuzumab resistance}.
\textit{BMC Systems Biology}, 3(1), 2009.

\tcitebullet Regina Samaga \textit{et al.}
\emphcolor{The Logic of EGFR/ErbB Signaling: Theoretical Properties and Analysis of High-Throughput Data}.
\textit{PLoS Computational Biology}, 5(8), 2009.

\tcitebullet Steffen Klamt \textit{et al.}
\emphcolor{A methodology for the structural and functional analysis of signaling and regulatory networks}.
\textit{BMC Bioinformatics}, 7(1), 2006.

\tcitebullet Julio Saez-Rodriguez \textit{et al.}
\emphcolor{A Logical Model Provides Insights into T Cell Receptor Signaling}.
\textit{PLoS Computational Biology}, 3(8), 2007.

\end{frame}

%\tcitebullet Adrien Richard, Jean-Paul Comet, Gilles Bernot. \textit{Modern Formal Methods and App.}, chapter \emphcolor{Formal Methods for Modeling Biological Regulatory Networks}, pages 83--122, 2006.

%\tcitebullet Maxime Folschette, Loïc Paulevé, Katsumi Inoue, Morgan Magnin, Olivier Roux. \emphcolor{Concretizing the Process Hitting into Biological Regulatory Networks}. In David Gilbert and Monika Heiner, editors, \textit{Computational Methods in Systems Biology X}, Lecture Notes in Computer Science, pages 166--186. Springer Berlin Heidelberg, 2012.

%\tcitebullet Maxime Folschette, Loïc Paulevé, Morgan Magnin, Olivier Roux. \emphcolor{Under-approximation of Reachability in Multivalued Asynchronous Networks}. In E. Merelli and A. Troina, editors, \textit{4th International Workshop on Interactions between Computer Science and Biology (CS2Bio’13)}, Electronic Notes in Theoretical Computer Science, Volume 299, 33–51. June 2013.

%\tcitebullet Loïc Paulevé. PhD thesis: \emphcolor{\textit{Modélisation, Simulation et Vérification des Grands Réseaux de Régulation Biologique}}, October 2011, Nantes, France.

%\tcitebullet Loïc Paulevé, Morgan Magnin, and Olivier Roux. \textit{Tuning Temporal Features within the Stochastic $\pi$-Calculus}. IEEE Transactions on Software Engineering, 37(6), pages 858--871, 2011.

%\tcitebullet Loïc Paulevé and Adrien Richard. \textit{Topological Fixed Points in Boolean Networks}. Comptes Rendus de l'Académie des Sciences - Series I - Mathematics, 348(15-16), pages 825--828, 2010.

%\tcitebullet Gilles Bernot, Franck Cassez, Jean-Paul Comet, Franck Delaplace, Céline Müller, Olivier Roux. \emphcolor{Semantics of Biological Regulatory Networks}. \textit{Proceedings of the First Workshop on Concurrent Models in Molecular Biology}, Electronic Notes in Theoretical Computer Science 180(3), pages 3--14, 2007.

%\tcitebullet Adrien Richard. \emphcolor{Negative circuits and sustained oscillations in asynchronous automata networks}. \textit{Advances in Applied Mathematics} 44(4), pages 378--392, 2010.


\section{Introduction of Stochastic Parameters}
% Propriétés stochastiques

%\newcommand{\stochaa}{{\footnotesize $\spadesuit$}}
%\newcommand{\stochab}{{\footnotesize $\clubsuit$}}
\newcommand{\stochaa}{A}
\newcommand{\stochab}{B}
\newcommand{\stochainf}{{\small $\infty$}}

\begin{frame}[c]
  \frametitle{Stochastic Parameters}
  \framesubtitle{\tcite{\cpmrtcsb}}

\begin{itemize}
  \item Introduction of temporal properties
  \item Stochastic parameters $(r,sa)$ equivalent to a \tval{firing interval} $[d;D]$
\end{itemize}

\bigskip
\begin{columns}
\begin{column}{0.5\textwidth}

\vspace*{-2em}
\scalebox{0.9}{
\begin{tikzpicture}[plot,xscale=0.35,yscale=1]
%  \path[draw,use as bounding box] (-.5,-.5) rectangle (16,.5);
  \draw[axe] (0,0) -- (15.5,0) node[right] {$t$};
  \draw[axe] (0,0) -- (0,1);
  \draw[ticks] (0,0) node[below] {$0$};
  \draw[mean] (4,0) -- (4,0.9) node[right]{$\frac{1}{r}$};
  \draw[dashed] (0.10127,0) -- (0.10127,0.9) node[right] {$d$} (14.75552,0) --
  (14.75552,0.9) node[left] {$D$};
  \pgfplothandlerlineto
  \pgfplotxyfile{plots/BioAtlanSTIC-0409/erlang-0.25-1.table}
  \pgfusepath{stroke}
  \draw[interval] (0.10127,0) -- (14.75552,0);
\end{tikzpicture}
}

\scalebox{0.9}{
\begin{tikzpicture}[plot,xscale=0.35,yscale=1]
  \draw[axe] (0,0) -- (15.5,0) node[right] {$t$};
  \draw[axe] (0,0) -- (0,1);
  \draw[ticks] (0,0) node[below] {$0$};
  \draw[mean] (4,0) -- (4,0.9) node[right]{$\frac{1}{r}$};
  \draw[dashed] (1.29879,0) -- (1.29879,0.9) node[right] {$d$} (8.19327,0) --
  (8.19327,0.9) node[left] {$D$};
  \pgfplothandlerlineto
  \pgfplotxyfile{plots/BioAtlanSTIC-0409/erlang-0.25-5.table}
  \pgfusepath{stroke}
  \draw[interval] (1.29879,0) -- (8.19327,0);
\end{tikzpicture}
}

\scalebox{0.9}{
\begin{tikzpicture}[plot,xscale=0.35,yscale=1]
  \draw[axe] (0,0) -- (15.5,0) node[right] {$t$};
  \draw[axe] (0,0) -- (0,1);
  \draw[ticks] (0,0) node[below] {$0$};
  \draw[mean] (4,0) -- (4,0.9) node[right]{$\frac{1}{r}$};
  \draw[dashed] (2.96888,0) -- (2.96888,0.9) node[left] {$d$} (5.18245,0) --
  (5.18245,0.9) node[right] {$D$};
  \pgfplothandlerlineto
  \pgfplotxyfile{plots/BioAtlanSTIC-0409/erlang-0.25-50.table}
  \pgfusepath{stroke}
  \draw[interval] (2.96888,0) -- (5.18245,0);
\end{tikzpicture}
}

~~\textcolor{red!70}{\rule{17mm}{4.5pt}} ~ Intervalle de tir

\end{column}
\begin{column}{0.5\textwidth}
\begin{center}

\uncover<2->{
\scalebox{0.9}{
\begin{tikzpicture}
  \path[use as bounding box] (-1,-1) rectangle (2,1);
  \TSort{(0,0)}{a}{2}{l}
  \TSort{(2,0)}{b}{2}{r}
  \THit{a_0}{}{b_0}{.west}{b_1}
  \THit{a_0}{out=-120,in=180,selfhit}{a_0}{.west}{a_1}
  \path[bounce]
  \TBounce{a_0}{bend left}{a_1}{.south}
  \TBounce{b_0}{bend left}{b_1}{.south}
  ;
  \TState{1-}{a_0,b_0}

  \node[labelprio2, labelstocha] at (-1.5,-0.5) {\stochaa};
  \node[labelprio3, labelstocha] at (1,0.25) {\stochab};

%  \fill[orange!70] node (1,0.2) circle (1ex) {$\diamond$};
%  \fill[red!70] (-1.5,-0.5) circle (1ex);
\end{tikzpicture}
}

\scalebox{0.9}{
\begin{tikzpicture}[plot]
\node[anchor=west] at (0,0.4) {$\PHfrappe{a_0}{b_0}{b_1}$ \enspace (\stochab)};
\draw[axe] (0,0) -- (4,0) node[right] {$t$};
\draw (0,0.1) -- (0,-0.1);
\draw[interval,orange] (1.7,0) -- (3,0);

\node[anchor=west] at (0,-0.6) {$\PHfrappe{a_0}{a_0}{a_1}$ \enspace (\stochaa)};
\draw[axe] (0,-1) -- (4,-1) node[right] {$t$};
\draw (0,-0.9) -- (0,-1.1);
\draw[interval] (0.2,-1) -- (1.5,-1);
\end{tikzpicture}
}

\noindent
\f \tval{Very low probability} to reach $b_1$
}

\end{center}
\end{column}
\end{columns}

\pause
\medskip
% \begin{fleches}
%   \item Tests by simulation
%   \item Model-checking
% \end{fleches}
\begin{itemize}
  \item Simulation \f not formal
  \item \textit{Model-checking} \f High complexity for an acceptable precision
\end{itemize}

\end{frame}

% Stochasticité dans Metazoan

\begin{frame}[t]
  \frametitle{Use of Stochastic Parameters}
  \framesubtitle{\tcite{\cpmrtcsb}}

\makenoprio

\begin{tikzpicture}
  \path[use as bounding box] (-1,0) rectangle (8,6.5);
  \exmetazoan
  
  \node[labelprio2, labelstocha] at (4,5.3) {\stochaa};
  \node[labelprio2, labelstocha] at (5.5,3.8) {\stochaa};

  \node[labelprio3, labelstocha] at (0,2.5) {\stochab};
  \node[labelprio3, labelstocha] at (0,6.3) {\stochab};

  \node[labelprio1, labelstocha] at (2.3,1.7) {\stochainf};
  \node[labelprio1, labelstocha] at (2.3,4.1) {\stochainf};
\end{tikzpicture}

\vspace*{-2.5cm}

\begin{flushright}

\scalebox{0.9}{
\begin{tikzpicture}[plot]
\node at (0,2.3) {\stochainf};
\draw[axe] (0,2) -- (4,2) node[right] {$t$};
\draw (0,1.9) -- (0,2.1);
\draw[interval,blue!50] (-0.1,2) -- (0.1,2);

\node at (0.85,1.3) {\stochaa};
\draw[axe] (0,1) -- (4,1) node[right] {$t$};
\draw (0,0.9) -- (0,1.1);
\draw[interval] (0.2,1) -- (1.5,1);

\node at (2.35,0.3) {\stochab};
\draw[axe] (0,0) -- (4,0) node[right] {$t$};
\draw (0,0.1) -- (0,-0.1);
\draw[interval,orange] (1.7,0) -- (3,0);

\end{tikzpicture}
}

\end{flushright}

\end{frame}



\begin{frame}[c]
  \frametitle{Temporal Simulation}
  \framesubtitle{\tcite{\paulevephd}}

\begin{itemize}
  \item Simulation with stochastic parameters:
\end{itemize}

\medskip

\scalebox{.8}{% GNUPLOT: LaTeX picture
\setlength{\unitlength}{0.240900pt}
\ifx\plotpoint\undefined\newsavebox{\plotpoint}\fi
\sbox{\plotpoint}{\rule[-0.200pt]{0.400pt}{0.400pt}}%
\begin{picture}(1440,188)(0,0)
\font\gnuplot=cmtt10 at 8pt
\gnuplot
\put(20,111){\makebox(0,0){$a$}}
\sbox{\plotpoint}{\rule[-0.200pt]{0.400pt}{0.400pt}}%
\put(80.0,66.0){\rule[-0.200pt]{4.818pt}{0.400pt}}
\put(64,66){\makebox(0,0)[r]{ 0}}
\put(1371.0,66.0){\rule[-0.200pt]{4.818pt}{0.400pt}}
\put(80.0,157.0){\rule[-0.200pt]{4.818pt}{0.400pt}}
\put(64,157){\makebox(0,0)[r]{ 1}}
\put(1371.0,157.0){\rule[-0.200pt]{4.818pt}{0.400pt}}
\put(80.0,66.0){\rule[-0.200pt]{0.400pt}{4.818pt}}
\put(80,33){\makebox(0,0){ 0}}
\put(80.0,137.0){\rule[-0.200pt]{0.400pt}{4.818pt}}
\put(244.0,66.0){\rule[-0.200pt]{0.400pt}{4.818pt}}
\put(244,33){\makebox(0,0){ 5}}
\put(244.0,137.0){\rule[-0.200pt]{0.400pt}{4.818pt}}
\put(408.0,66.0){\rule[-0.200pt]{0.400pt}{4.818pt}}
\put(408,33){\makebox(0,0){ 10}}
\put(408.0,137.0){\rule[-0.200pt]{0.400pt}{4.818pt}}
\put(572.0,66.0){\rule[-0.200pt]{0.400pt}{4.818pt}}
\put(572,33){\makebox(0,0){ 15}}
\put(572.0,137.0){\rule[-0.200pt]{0.400pt}{4.818pt}}
\put(735.0,66.0){\rule[-0.200pt]{0.400pt}{4.818pt}}
\put(735,33){\makebox(0,0){ 20}}
\put(735.0,137.0){\rule[-0.200pt]{0.400pt}{4.818pt}}
\put(899.0,66.0){\rule[-0.200pt]{0.400pt}{4.818pt}}
\put(899,33){\makebox(0,0){ 25}}
\put(899.0,137.0){\rule[-0.200pt]{0.400pt}{4.818pt}}
\put(1063.0,66.0){\rule[-0.200pt]{0.400pt}{4.818pt}}
\put(1063,33){\makebox(0,0){ 30}}
\put(1063.0,137.0){\rule[-0.200pt]{0.400pt}{4.818pt}}
\put(1227.0,66.0){\rule[-0.200pt]{0.400pt}{4.818pt}}
\put(1227,33){\makebox(0,0){ 35}}
\put(1227.0,137.0){\rule[-0.200pt]{0.400pt}{4.818pt}}
\put(1391.0,66.0){\rule[-0.200pt]{0.400pt}{4.818pt}}
\put(1391,33){\makebox(0,0){ 40}}
\put(1391.0,137.0){\rule[-0.200pt]{0.400pt}{4.818pt}}
\put(80,66){\usebox{\plotpoint}}
\put(80.0,66.0){\rule[-0.200pt]{9.636pt}{0.400pt}}
\put(120.0,66.0){\rule[-0.200pt]{0.400pt}{21.922pt}}
\put(120.0,157.0){\rule[-0.200pt]{14.213pt}{0.400pt}}
\put(179.0,66.0){\rule[-0.200pt]{0.400pt}{21.922pt}}
\put(179.0,66.0){\rule[-0.200pt]{15.899pt}{0.400pt}}
\put(245.0,66.0){\rule[-0.200pt]{0.400pt}{21.922pt}}
\put(245.0,157.0){\rule[-0.200pt]{12.045pt}{0.400pt}}
\put(295.0,66.0){\rule[-0.200pt]{0.400pt}{21.922pt}}
\put(295.0,66.0){\rule[-0.200pt]{19.272pt}{0.400pt}}
\put(375.0,66.0){\rule[-0.200pt]{0.400pt}{21.922pt}}
\put(375.0,157.0){\rule[-0.200pt]{12.045pt}{0.400pt}}
\put(425.0,66.0){\rule[-0.200pt]{0.400pt}{21.922pt}}
\put(425.0,66.0){\rule[-0.200pt]{18.549pt}{0.400pt}}
\put(502.0,66.0){\rule[-0.200pt]{0.400pt}{21.922pt}}
\put(502.0,157.0){\rule[-0.200pt]{15.658pt}{0.400pt}}
\put(567.0,66.0){\rule[-0.200pt]{0.400pt}{21.922pt}}
\put(567.0,66.0){\rule[-0.200pt]{10.600pt}{0.400pt}}
\put(611.0,66.0){\rule[-0.200pt]{0.400pt}{21.922pt}}
\put(611.0,157.0){\rule[-0.200pt]{13.972pt}{0.400pt}}
\put(669.0,66.0){\rule[-0.200pt]{0.400pt}{21.922pt}}
\put(669.0,66.0){\rule[-0.200pt]{17.345pt}{0.400pt}}
\put(741.0,66.0){\rule[-0.200pt]{0.400pt}{21.922pt}}
\put(741.0,157.0){\rule[-0.200pt]{17.586pt}{0.400pt}}
\put(814.0,66.0){\rule[-0.200pt]{0.400pt}{21.922pt}}
\put(814.0,66.0){\rule[-0.200pt]{15.658pt}{0.400pt}}
\put(879.0,66.0){\rule[-0.200pt]{0.400pt}{21.922pt}}
\put(879.0,157.0){\rule[-0.200pt]{18.549pt}{0.400pt}}
\put(956.0,66.0){\rule[-0.200pt]{0.400pt}{21.922pt}}
\put(956.0,66.0){\rule[-0.200pt]{16.622pt}{0.400pt}}
\put(1025.0,66.0){\rule[-0.200pt]{0.400pt}{21.922pt}}
\put(1025.0,157.0){\rule[-0.200pt]{88.169pt}{0.400pt}}
\end{picture}
}
\scalebox{.8}{% GNUPLOT: LaTeX picture
\setlength{\unitlength}{0.240900pt}
\ifx\plotpoint\undefined\newsavebox{\plotpoint}\fi
\begin{picture}(1440,188)(0,0)
\font\gnuplot=cmtt10 at 8pt
\gnuplot
\put(20,111){\makebox(0,0){$c$}}
\sbox{\plotpoint}{\rule[-0.200pt]{0.400pt}{0.400pt}}%
\put(80.0,66.0){\rule[-0.200pt]{4.818pt}{0.400pt}}
\put(64,66){\makebox(0,0)[r]{ 0}}
\put(1371.0,66.0){\rule[-0.200pt]{4.818pt}{0.400pt}}
\put(80.0,157.0){\rule[-0.200pt]{4.818pt}{0.400pt}}
\put(64,157){\makebox(0,0)[r]{ 1}}
\put(1371.0,157.0){\rule[-0.200pt]{4.818pt}{0.400pt}}
\put(80.0,66.0){\rule[-0.200pt]{0.400pt}{4.818pt}}
\put(80,33){\makebox(0,0){ 0}}
\put(80.0,137.0){\rule[-0.200pt]{0.400pt}{4.818pt}}
\put(244.0,66.0){\rule[-0.200pt]{0.400pt}{4.818pt}}
\put(244,33){\makebox(0,0){ 5}}
\put(244.0,137.0){\rule[-0.200pt]{0.400pt}{4.818pt}}
\put(408.0,66.0){\rule[-0.200pt]{0.400pt}{4.818pt}}
\put(408,33){\makebox(0,0){ 10}}
\put(408.0,137.0){\rule[-0.200pt]{0.400pt}{4.818pt}}
\put(572.0,66.0){\rule[-0.200pt]{0.400pt}{4.818pt}}
\put(572,33){\makebox(0,0){ 15}}
\put(572.0,137.0){\rule[-0.200pt]{0.400pt}{4.818pt}}
\put(735.0,66.0){\rule[-0.200pt]{0.400pt}{4.818pt}}
\put(735,33){\makebox(0,0){ 20}}
\put(735.0,137.0){\rule[-0.200pt]{0.400pt}{4.818pt}}
\put(899.0,66.0){\rule[-0.200pt]{0.400pt}{4.818pt}}
\put(899,33){\makebox(0,0){ 25}}
\put(899.0,137.0){\rule[-0.200pt]{0.400pt}{4.818pt}}
\put(1063.0,66.0){\rule[-0.200pt]{0.400pt}{4.818pt}}
\put(1063,33){\makebox(0,0){ 30}}
\put(1063.0,137.0){\rule[-0.200pt]{0.400pt}{4.818pt}}
\put(1227.0,66.0){\rule[-0.200pt]{0.400pt}{4.818pt}}
\put(1227,33){\makebox(0,0){ 35}}
\put(1227.0,137.0){\rule[-0.200pt]{0.400pt}{4.818pt}}
\put(1391.0,66.0){\rule[-0.200pt]{0.400pt}{4.818pt}}
\put(1391,33){\makebox(0,0){ 40}}
\put(1391.0,137.0){\rule[-0.200pt]{0.400pt}{4.818pt}}
\put(80,66){\usebox{\plotpoint}}
\put(80.0,66.0){\rule[-0.200pt]{16.622pt}{0.400pt}}
\put(149.0,66.0){\rule[-0.200pt]{0.400pt}{21.922pt}}
\put(149.0,157.0){\rule[-0.200pt]{14.454pt}{0.400pt}}
\put(209.0,66.0){\rule[-0.200pt]{0.400pt}{21.922pt}}
\put(209.0,66.0){\rule[-0.200pt]{13.490pt}{0.400pt}}
\put(265.0,66.0){\rule[-0.200pt]{0.400pt}{21.922pt}}
\put(265.0,157.0){\rule[-0.200pt]{14.936pt}{0.400pt}}
\put(327.0,66.0){\rule[-0.200pt]{0.400pt}{21.922pt}}
\put(327.0,66.0){\rule[-0.200pt]{16.140pt}{0.400pt}}
\put(394.0,66.0){\rule[-0.200pt]{0.400pt}{21.922pt}}
\put(394.0,157.0){\rule[-0.200pt]{18.308pt}{0.400pt}}
\put(470.0,66.0){\rule[-0.200pt]{0.400pt}{21.922pt}}
\put(470.0,66.0){\rule[-0.200pt]{14.936pt}{0.400pt}}
\put(532.0,66.0){\rule[-0.200pt]{0.400pt}{21.922pt}}
\put(532.0,157.0){\rule[-0.200pt]{12.045pt}{0.400pt}}
\put(582.0,66.0){\rule[-0.200pt]{0.400pt}{21.922pt}}
\put(582.0,66.0){\rule[-0.200pt]{13.731pt}{0.400pt}}
\put(639.0,66.0){\rule[-0.200pt]{0.400pt}{21.922pt}}
\put(639.0,157.0){\rule[-0.200pt]{16.622pt}{0.400pt}}
\put(708.0,66.0){\rule[-0.200pt]{0.400pt}{21.922pt}}
\put(708.0,66.0){\rule[-0.200pt]{17.345pt}{0.400pt}}
\put(780.0,66.0){\rule[-0.200pt]{0.400pt}{21.922pt}}
\put(780.0,157.0){\rule[-0.200pt]{14.936pt}{0.400pt}}
\put(842.0,66.0){\rule[-0.200pt]{0.400pt}{21.922pt}}
\put(842.0,66.0){\rule[-0.200pt]{19.995pt}{0.400pt}}
\put(925.0,66.0){\rule[-0.200pt]{0.400pt}{21.922pt}}
\put(925.0,157.0){\rule[-0.200pt]{15.899pt}{0.400pt}}
\put(991.0,66.0){\rule[-0.200pt]{0.400pt}{21.922pt}}
\put(991.0,66.0){\rule[-0.200pt]{17.345pt}{0.400pt}}
\put(1063.0,66.0){\rule[-0.200pt]{0.400pt}{21.922pt}}
\put(1063.0,157.0){\rule[-0.200pt]{4.336pt}{0.400pt}}
\put(1081.0,66.0){\rule[-0.200pt]{0.400pt}{21.922pt}}
\put(1081.0,66.0){\rule[-0.200pt]{74.679pt}{0.400pt}}
\end{picture}
}
\scalebox{.8}{% GNUPLOT: LaTeX picture
\setlength{\unitlength}{0.240900pt}
\ifx\plotpoint\undefined\newsavebox{\plotpoint}\fi
\begin{picture}(1440,188)(0,0)
\font\gnuplot=cmtt10 at 8pt
\gnuplot
\put(20,111){\makebox(0,0){$f$}}
\sbox{\plotpoint}{\rule[-0.200pt]{0.400pt}{0.400pt}}%
\put(80.0,66.0){\rule[-0.200pt]{4.818pt}{0.400pt}}
\put(64,66){\makebox(0,0)[r]{ 0}}
\put(1371.0,66.0){\rule[-0.200pt]{4.818pt}{0.400pt}}
\put(80.0,157.0){\rule[-0.200pt]{4.818pt}{0.400pt}}
\put(64,157){\makebox(0,0)[r]{ 1}}
\put(1371.0,157.0){\rule[-0.200pt]{4.818pt}{0.400pt}}
\put(80.0,66.0){\rule[-0.200pt]{0.400pt}{4.818pt}}
\put(80,33){\makebox(0,0){ 0}}
\put(80.0,137.0){\rule[-0.200pt]{0.400pt}{4.818pt}}
\put(244.0,66.0){\rule[-0.200pt]{0.400pt}{4.818pt}}
\put(244,33){\makebox(0,0){ 5}}
\put(244.0,137.0){\rule[-0.200pt]{0.400pt}{4.818pt}}
\put(408.0,66.0){\rule[-0.200pt]{0.400pt}{4.818pt}}
\put(408,33){\makebox(0,0){ 10}}
\put(408.0,137.0){\rule[-0.200pt]{0.400pt}{4.818pt}}
\put(572.0,66.0){\rule[-0.200pt]{0.400pt}{4.818pt}}
\put(572,33){\makebox(0,0){ 15}}
\put(572.0,137.0){\rule[-0.200pt]{0.400pt}{4.818pt}}
\put(735.0,66.0){\rule[-0.200pt]{0.400pt}{4.818pt}}
\put(735,33){\makebox(0,0){ 20}}
\put(735.0,137.0){\rule[-0.200pt]{0.400pt}{4.818pt}}
\put(899.0,66.0){\rule[-0.200pt]{0.400pt}{4.818pt}}
\put(899,33){\makebox(0,0){ 25}}
\put(899.0,137.0){\rule[-0.200pt]{0.400pt}{4.818pt}}
\put(1063.0,66.0){\rule[-0.200pt]{0.400pt}{4.818pt}}
\put(1063,33){\makebox(0,0){ 30}}
\put(1063.0,137.0){\rule[-0.200pt]{0.400pt}{4.818pt}}
\put(1227.0,66.0){\rule[-0.200pt]{0.400pt}{4.818pt}}
\put(1227,33){\makebox(0,0){ 35}}
\put(1227.0,137.0){\rule[-0.200pt]{0.400pt}{4.818pt}}
\put(1391.0,66.0){\rule[-0.200pt]{0.400pt}{4.818pt}}
\put(1391,33){\makebox(0,0){ 40}}
\put(1391.0,137.0){\rule[-0.200pt]{0.400pt}{4.818pt}}
\put(80,157){\usebox{\plotpoint}}
\put(80.0,157.0){\rule[-0.200pt]{235.841pt}{0.400pt}}
\put(1059.0,66.0){\rule[-0.200pt]{0.400pt}{21.922pt}}
\put(1059.0,66.0){\rule[-0.200pt]{79.979pt}{0.400pt}}
\end{picture}
}

\bigskip
\bigskip

\begin{itemize}
  \item Other possible analysis: stochastic model checkers (PRISM)
  \begin{itemize}
    \item[\f] But combinatoric explosion: PRISM fails for more than 5 components
  \end{itemize}

\end{itemize}

\end{frame}

% Priorités dans Metazoan

\begin{frame}[t]
  \frametitle{Use of the Classes of Priorities}
  \framesubtitle{\tcite{\cfpmrcsbio}}

\makenoprio

\begin{tikzpicture}
  \path[use as bounding box] (-2,0) rectangle (8,6.5);
  \exmetazoan

  \node[labelprio1] at (2.3,4) {$1$};
  \node[labelprio1] at (2.6,0.8) {$1$};
  \node[labelprio2] at (5.5,3.9) {$2$};
  \node[labelprio2] at (3.5,5.3) {$2$};
  \node[labelprio3] at (0,2.5) {$3$};
  \node[labelprio3] at (0.8,5.8) {$3$};
  %\node[labelprio4] at (1.5,1.8) {$4$};
  
  \TState{3,9,15}{f_1, a_0, c_0, fc_2}
  \TState{4,10}{f_1, a_1, c_0, fc_2}
  \TState{5,11}{f_1, a_1, c_1, fc_2}
  \TState{6,12}{f_1, a_1, c_1, fc_3}
  \TState{7,13}{f_1, a_0, c_1, fc_3}
  \TState{8,14}{f_1, a_0, c_0, fc_3}
\end{tikzpicture}

\pause[2]
\vspace*{-2.5cm}
\hfill
\begin{tikzpicture}
  \tikz \foreach \x in {0,...,12}
    \draw[dotted] (\x/4,0) -- (\x/4,1.5);

  \draw[dotted] (0,0) -- (-3,0);
  \draw[dotted] (0,1.5) -- (-3,1.5);

  \only<4->{\fill (-3,0) rectangle (-2.75,1.5);}
  \only<5->{\fill (-2.75,0) rectangle (-2.5,1.5);}
  \only<6->{\fill (-2.5,0) rectangle (-2.25,1.5);}
  \only<7->{\fill[gray!30] (-2.25,0) rectangle (-2,1.5);}
  \only<8->{\fill[gray!30] (-2,0) rectangle (-1.75,1.5);}
  \only<9->{\fill[gray!30] (-1.75,0) rectangle (-1.5,1.5);}
  \only<10->{\fill (-1.5,0) rectangle (-1.25,1.5);}
  \only<11->{\fill (-1.25,0) rectangle (-1,1.5);}
  \only<12->{\fill (-1,0) rectangle (-0.75,1.5);}
  \only<13->{\fill[gray!30] (-0.75,0) rectangle (-0.5,1.5);}
  \only<14->{\fill[gray!30] (-0.5,0) rectangle (-0.25,1.5);}
  \only<15->{\fill[gray!30] (-0.25,0) rectangle (0,1.5);}
\end{tikzpicture}

\pause[15]
%\vspace*{.3cm}
\bigskip
\begin{flushright}
  \f Only one possible stationary behavior
\end{flushright}

\end{frame}



\begin{frame}[c]
  \frametitle{Abstraction of Temporal Parameters}
  \framesubtitle{\tcite{\paulevephd}}

\begin{itemize}
  \item Simulation with stochastic parameters:
\end{itemize}

\medskip

\scalebox{.8}{% GNUPLOT: LaTeX picture
\setlength{\unitlength}{0.240900pt}
\ifx\plotpoint\undefined\newsavebox{\plotpoint}\fi
\sbox{\plotpoint}{\rule[-0.200pt]{0.400pt}{0.400pt}}%
\begin{picture}(1440,188)(0,0)
\font\gnuplot=cmtt10 at 8pt
\gnuplot
\put(20,111){\makebox(0,0){$a$}}
\sbox{\plotpoint}{\rule[-0.200pt]{0.400pt}{0.400pt}}%
\put(80.0,66.0){\rule[-0.200pt]{4.818pt}{0.400pt}}
\put(64,66){\makebox(0,0)[r]{ 0}}
\put(1371.0,66.0){\rule[-0.200pt]{4.818pt}{0.400pt}}
\put(80.0,157.0){\rule[-0.200pt]{4.818pt}{0.400pt}}
\put(64,157){\makebox(0,0)[r]{ 1}}
\put(1371.0,157.0){\rule[-0.200pt]{4.818pt}{0.400pt}}
\put(80.0,66.0){\rule[-0.200pt]{0.400pt}{4.818pt}}
\put(80,33){\makebox(0,0){ 0}}
\put(80.0,137.0){\rule[-0.200pt]{0.400pt}{4.818pt}}
\put(244.0,66.0){\rule[-0.200pt]{0.400pt}{4.818pt}}
\put(244,33){\makebox(0,0){ 5}}
\put(244.0,137.0){\rule[-0.200pt]{0.400pt}{4.818pt}}
\put(408.0,66.0){\rule[-0.200pt]{0.400pt}{4.818pt}}
\put(408,33){\makebox(0,0){ 10}}
\put(408.0,137.0){\rule[-0.200pt]{0.400pt}{4.818pt}}
\put(572.0,66.0){\rule[-0.200pt]{0.400pt}{4.818pt}}
\put(572,33){\makebox(0,0){ 15}}
\put(572.0,137.0){\rule[-0.200pt]{0.400pt}{4.818pt}}
\put(735.0,66.0){\rule[-0.200pt]{0.400pt}{4.818pt}}
\put(735,33){\makebox(0,0){ 20}}
\put(735.0,137.0){\rule[-0.200pt]{0.400pt}{4.818pt}}
\put(899.0,66.0){\rule[-0.200pt]{0.400pt}{4.818pt}}
\put(899,33){\makebox(0,0){ 25}}
\put(899.0,137.0){\rule[-0.200pt]{0.400pt}{4.818pt}}
\put(1063.0,66.0){\rule[-0.200pt]{0.400pt}{4.818pt}}
\put(1063,33){\makebox(0,0){ 30}}
\put(1063.0,137.0){\rule[-0.200pt]{0.400pt}{4.818pt}}
\put(1227.0,66.0){\rule[-0.200pt]{0.400pt}{4.818pt}}
\put(1227,33){\makebox(0,0){ 35}}
\put(1227.0,137.0){\rule[-0.200pt]{0.400pt}{4.818pt}}
\put(1391.0,66.0){\rule[-0.200pt]{0.400pt}{4.818pt}}
\put(1391,33){\makebox(0,0){ 40}}
\put(1391.0,137.0){\rule[-0.200pt]{0.400pt}{4.818pt}}
\put(80,66){\usebox{\plotpoint}}
\put(80.0,66.0){\rule[-0.200pt]{9.636pt}{0.400pt}}
\put(120.0,66.0){\rule[-0.200pt]{0.400pt}{21.922pt}}
\put(120.0,157.0){\rule[-0.200pt]{14.213pt}{0.400pt}}
\put(179.0,66.0){\rule[-0.200pt]{0.400pt}{21.922pt}}
\put(179.0,66.0){\rule[-0.200pt]{15.899pt}{0.400pt}}
\put(245.0,66.0){\rule[-0.200pt]{0.400pt}{21.922pt}}
\put(245.0,157.0){\rule[-0.200pt]{12.045pt}{0.400pt}}
\put(295.0,66.0){\rule[-0.200pt]{0.400pt}{21.922pt}}
\put(295.0,66.0){\rule[-0.200pt]{19.272pt}{0.400pt}}
\put(375.0,66.0){\rule[-0.200pt]{0.400pt}{21.922pt}}
\put(375.0,157.0){\rule[-0.200pt]{12.045pt}{0.400pt}}
\put(425.0,66.0){\rule[-0.200pt]{0.400pt}{21.922pt}}
\put(425.0,66.0){\rule[-0.200pt]{18.549pt}{0.400pt}}
\put(502.0,66.0){\rule[-0.200pt]{0.400pt}{21.922pt}}
\put(502.0,157.0){\rule[-0.200pt]{15.658pt}{0.400pt}}
\put(567.0,66.0){\rule[-0.200pt]{0.400pt}{21.922pt}}
\put(567.0,66.0){\rule[-0.200pt]{10.600pt}{0.400pt}}
\put(611.0,66.0){\rule[-0.200pt]{0.400pt}{21.922pt}}
\put(611.0,157.0){\rule[-0.200pt]{13.972pt}{0.400pt}}
\put(669.0,66.0){\rule[-0.200pt]{0.400pt}{21.922pt}}
\put(669.0,66.0){\rule[-0.200pt]{17.345pt}{0.400pt}}
\put(741.0,66.0){\rule[-0.200pt]{0.400pt}{21.922pt}}
\put(741.0,157.0){\rule[-0.200pt]{17.586pt}{0.400pt}}
\put(814.0,66.0){\rule[-0.200pt]{0.400pt}{21.922pt}}
\put(814.0,66.0){\rule[-0.200pt]{15.658pt}{0.400pt}}
\put(879.0,66.0){\rule[-0.200pt]{0.400pt}{21.922pt}}
\put(879.0,157.0){\rule[-0.200pt]{18.549pt}{0.400pt}}
\put(956.0,66.0){\rule[-0.200pt]{0.400pt}{21.922pt}}
\put(956.0,66.0){\rule[-0.200pt]{16.622pt}{0.400pt}}
\put(1025.0,66.0){\rule[-0.200pt]{0.400pt}{21.922pt}}
\put(1025.0,157.0){\rule[-0.200pt]{88.169pt}{0.400pt}}
\end{picture}
}
\scalebox{.8}{% GNUPLOT: LaTeX picture
\setlength{\unitlength}{0.240900pt}
\ifx\plotpoint\undefined\newsavebox{\plotpoint}\fi
\begin{picture}(1440,188)(0,0)
\font\gnuplot=cmtt10 at 8pt
\gnuplot
\put(20,111){\makebox(0,0){$c$}}
\sbox{\plotpoint}{\rule[-0.200pt]{0.400pt}{0.400pt}}%
\put(80.0,66.0){\rule[-0.200pt]{4.818pt}{0.400pt}}
\put(64,66){\makebox(0,0)[r]{ 0}}
\put(1371.0,66.0){\rule[-0.200pt]{4.818pt}{0.400pt}}
\put(80.0,157.0){\rule[-0.200pt]{4.818pt}{0.400pt}}
\put(64,157){\makebox(0,0)[r]{ 1}}
\put(1371.0,157.0){\rule[-0.200pt]{4.818pt}{0.400pt}}
\put(80.0,66.0){\rule[-0.200pt]{0.400pt}{4.818pt}}
\put(80,33){\makebox(0,0){ 0}}
\put(80.0,137.0){\rule[-0.200pt]{0.400pt}{4.818pt}}
\put(244.0,66.0){\rule[-0.200pt]{0.400pt}{4.818pt}}
\put(244,33){\makebox(0,0){ 5}}
\put(244.0,137.0){\rule[-0.200pt]{0.400pt}{4.818pt}}
\put(408.0,66.0){\rule[-0.200pt]{0.400pt}{4.818pt}}
\put(408,33){\makebox(0,0){ 10}}
\put(408.0,137.0){\rule[-0.200pt]{0.400pt}{4.818pt}}
\put(572.0,66.0){\rule[-0.200pt]{0.400pt}{4.818pt}}
\put(572,33){\makebox(0,0){ 15}}
\put(572.0,137.0){\rule[-0.200pt]{0.400pt}{4.818pt}}
\put(735.0,66.0){\rule[-0.200pt]{0.400pt}{4.818pt}}
\put(735,33){\makebox(0,0){ 20}}
\put(735.0,137.0){\rule[-0.200pt]{0.400pt}{4.818pt}}
\put(899.0,66.0){\rule[-0.200pt]{0.400pt}{4.818pt}}
\put(899,33){\makebox(0,0){ 25}}
\put(899.0,137.0){\rule[-0.200pt]{0.400pt}{4.818pt}}
\put(1063.0,66.0){\rule[-0.200pt]{0.400pt}{4.818pt}}
\put(1063,33){\makebox(0,0){ 30}}
\put(1063.0,137.0){\rule[-0.200pt]{0.400pt}{4.818pt}}
\put(1227.0,66.0){\rule[-0.200pt]{0.400pt}{4.818pt}}
\put(1227,33){\makebox(0,0){ 35}}
\put(1227.0,137.0){\rule[-0.200pt]{0.400pt}{4.818pt}}
\put(1391.0,66.0){\rule[-0.200pt]{0.400pt}{4.818pt}}
\put(1391,33){\makebox(0,0){ 40}}
\put(1391.0,137.0){\rule[-0.200pt]{0.400pt}{4.818pt}}
\put(80,66){\usebox{\plotpoint}}
\put(80.0,66.0){\rule[-0.200pt]{16.622pt}{0.400pt}}
\put(149.0,66.0){\rule[-0.200pt]{0.400pt}{21.922pt}}
\put(149.0,157.0){\rule[-0.200pt]{14.454pt}{0.400pt}}
\put(209.0,66.0){\rule[-0.200pt]{0.400pt}{21.922pt}}
\put(209.0,66.0){\rule[-0.200pt]{13.490pt}{0.400pt}}
\put(265.0,66.0){\rule[-0.200pt]{0.400pt}{21.922pt}}
\put(265.0,157.0){\rule[-0.200pt]{14.936pt}{0.400pt}}
\put(327.0,66.0){\rule[-0.200pt]{0.400pt}{21.922pt}}
\put(327.0,66.0){\rule[-0.200pt]{16.140pt}{0.400pt}}
\put(394.0,66.0){\rule[-0.200pt]{0.400pt}{21.922pt}}
\put(394.0,157.0){\rule[-0.200pt]{18.308pt}{0.400pt}}
\put(470.0,66.0){\rule[-0.200pt]{0.400pt}{21.922pt}}
\put(470.0,66.0){\rule[-0.200pt]{14.936pt}{0.400pt}}
\put(532.0,66.0){\rule[-0.200pt]{0.400pt}{21.922pt}}
\put(532.0,157.0){\rule[-0.200pt]{12.045pt}{0.400pt}}
\put(582.0,66.0){\rule[-0.200pt]{0.400pt}{21.922pt}}
\put(582.0,66.0){\rule[-0.200pt]{13.731pt}{0.400pt}}
\put(639.0,66.0){\rule[-0.200pt]{0.400pt}{21.922pt}}
\put(639.0,157.0){\rule[-0.200pt]{16.622pt}{0.400pt}}
\put(708.0,66.0){\rule[-0.200pt]{0.400pt}{21.922pt}}
\put(708.0,66.0){\rule[-0.200pt]{17.345pt}{0.400pt}}
\put(780.0,66.0){\rule[-0.200pt]{0.400pt}{21.922pt}}
\put(780.0,157.0){\rule[-0.200pt]{14.936pt}{0.400pt}}
\put(842.0,66.0){\rule[-0.200pt]{0.400pt}{21.922pt}}
\put(842.0,66.0){\rule[-0.200pt]{19.995pt}{0.400pt}}
\put(925.0,66.0){\rule[-0.200pt]{0.400pt}{21.922pt}}
\put(925.0,157.0){\rule[-0.200pt]{15.899pt}{0.400pt}}
\put(991.0,66.0){\rule[-0.200pt]{0.400pt}{21.922pt}}
\put(991.0,66.0){\rule[-0.200pt]{17.345pt}{0.400pt}}
\put(1063.0,66.0){\rule[-0.200pt]{0.400pt}{21.922pt}}
\put(1063.0,157.0){\rule[-0.200pt]{4.336pt}{0.400pt}}
\put(1081.0,66.0){\rule[-0.200pt]{0.400pt}{21.922pt}}
\put(1081.0,66.0){\rule[-0.200pt]{74.679pt}{0.400pt}}
\end{picture}
}
\scalebox{.8}{% GNUPLOT: LaTeX picture
\setlength{\unitlength}{0.240900pt}
\ifx\plotpoint\undefined\newsavebox{\plotpoint}\fi
\begin{picture}(1440,188)(0,0)
\font\gnuplot=cmtt10 at 8pt
\gnuplot
\put(20,111){\makebox(0,0){$f$}}
\sbox{\plotpoint}{\rule[-0.200pt]{0.400pt}{0.400pt}}%
\put(80.0,66.0){\rule[-0.200pt]{4.818pt}{0.400pt}}
\put(64,66){\makebox(0,0)[r]{ 0}}
\put(1371.0,66.0){\rule[-0.200pt]{4.818pt}{0.400pt}}
\put(80.0,157.0){\rule[-0.200pt]{4.818pt}{0.400pt}}
\put(64,157){\makebox(0,0)[r]{ 1}}
\put(1371.0,157.0){\rule[-0.200pt]{4.818pt}{0.400pt}}
\put(80.0,66.0){\rule[-0.200pt]{0.400pt}{4.818pt}}
\put(80,33){\makebox(0,0){ 0}}
\put(80.0,137.0){\rule[-0.200pt]{0.400pt}{4.818pt}}
\put(244.0,66.0){\rule[-0.200pt]{0.400pt}{4.818pt}}
\put(244,33){\makebox(0,0){ 5}}
\put(244.0,137.0){\rule[-0.200pt]{0.400pt}{4.818pt}}
\put(408.0,66.0){\rule[-0.200pt]{0.400pt}{4.818pt}}
\put(408,33){\makebox(0,0){ 10}}
\put(408.0,137.0){\rule[-0.200pt]{0.400pt}{4.818pt}}
\put(572.0,66.0){\rule[-0.200pt]{0.400pt}{4.818pt}}
\put(572,33){\makebox(0,0){ 15}}
\put(572.0,137.0){\rule[-0.200pt]{0.400pt}{4.818pt}}
\put(735.0,66.0){\rule[-0.200pt]{0.400pt}{4.818pt}}
\put(735,33){\makebox(0,0){ 20}}
\put(735.0,137.0){\rule[-0.200pt]{0.400pt}{4.818pt}}
\put(899.0,66.0){\rule[-0.200pt]{0.400pt}{4.818pt}}
\put(899,33){\makebox(0,0){ 25}}
\put(899.0,137.0){\rule[-0.200pt]{0.400pt}{4.818pt}}
\put(1063.0,66.0){\rule[-0.200pt]{0.400pt}{4.818pt}}
\put(1063,33){\makebox(0,0){ 30}}
\put(1063.0,137.0){\rule[-0.200pt]{0.400pt}{4.818pt}}
\put(1227.0,66.0){\rule[-0.200pt]{0.400pt}{4.818pt}}
\put(1227,33){\makebox(0,0){ 35}}
\put(1227.0,137.0){\rule[-0.200pt]{0.400pt}{4.818pt}}
\put(1391.0,66.0){\rule[-0.200pt]{0.400pt}{4.818pt}}
\put(1391,33){\makebox(0,0){ 40}}
\put(1391.0,137.0){\rule[-0.200pt]{0.400pt}{4.818pt}}
\put(80,157){\usebox{\plotpoint}}
\put(80.0,157.0){\rule[-0.200pt]{235.841pt}{0.400pt}}
\put(1059.0,66.0){\rule[-0.200pt]{0.400pt}{21.922pt}}
\put(1059.0,66.0){\rule[-0.200pt]{79.979pt}{0.400pt}}
\end{picture}
}

\bigskip
\bigskip

\begin{itemize}
  \item Other possible analysis: stochastic model checkers (PRISM)
  \begin{itemize}
    \item[\f] But combinatoric explosion: PRISM fails for more than 5 components
  \end{itemize}

\end{itemize}

\end{frame}



\begin{frame}[t]
  \frametitle{Addition of classes of priorities}
  \framesubtitle{\tcite{\cfpmrcsbio}}

\bigskip
\begin{itemize}
  \item Each action is associated to a discrete priority
  \item An action is playable only if no other action with higher priority is playable
\end{itemize}

\medskip

\begin{center}

\begin{tabular}{*{5}{>{\centering}p{1cm}}}
  \tikz \node[labelprio1] {$1$}; &
  \tikz \node[labelprio2] {$2$}; &
  \tikz \node[labelprio3] {$3$}; &
  \raisebox{5pt}{\ldots} &
  \tikz \node[labelprion] {$n$};
\vspace*{.5em} \tabularnewline \hline
  \multicolumn{2}{l}{\parbox{1.5cm}{\vspace*{.5em}highest\\priority}} &&
  \multicolumn{2}{r}{\parbox{1.5cm}{\raggedleft\vspace*{.5em}lowest\\priority}}
\end{tabular}
\hspace*{-1em}
\raisebox{2.2pt}{$\blacktriangleright$}

\bigskip
\end{center}

\begin{itemize}
  \item Allow to model classes of actions with similar speeds or temporal parameters
\end{itemize}
\begin{center}
\begin{tabular}{*{5}{>{\centering}p{1cm}}}
  \tikz \node[labelprio1,labelstocha] {$A$}; &
  \tikz \node[labelprio2,labelstocha] {$B$}; &
  \tikz \node[labelprio3,labelstocha] {$C$}; &
  \raisebox{5pt}{\ldots} &
  \tikz \node[labelprion,labelstocha] {$N$};
\vspace*{.5em} \tabularnewline \hline
  \multicolumn{1}{r}{\parbox{1cm}{\hspace*{-1.7cm}\parbox{2.5cm}{\raggedleft\vspace*{.5em}\tval{instantaneous}}}} &
  \multicolumn{2}{l}{\parbox{2cm}{\vspace*{.5em}\tval{very fast}}} &
  \multicolumn{2}{r}{\parbox{2cm}{\raggedleft\vspace*{.5em}\tval{very slow}}}
\end{tabular}
\hspace*{-1em}
\raisebox{-.4em}{$\blacktriangleright$}
\end{center}

\end{frame}



\begin{frame}[t]
  \frametitle{Limitation of the Classes of Priorities}
  \framesubtitle{\tcite{\cfpmrcsbio}}

\makenoprio

\begin{tikzpicture}
  \path[use as bounding box] (-2,0) rectangle (8,6.5);
  \exmetazoan
  
  \only<2>{
    \THit{f_1.north east}{selfhit, min distance=30, bend left, out=150, in=90,hlr}{f_1}{.south east}{f_0}
    \path[bounce, bend left=50]
      \TBounce{f_1}{hlr}{f_0}{.north east};
  }
  
  \node[labelprio1] at (2.3,4) {$1$};
  \node[labelprio1] at (2.6,0.8) {$1$};
  \node[labelprio2] at (5.5,3.9) {$2$};
  \node[labelprio2] at (3.5,5.3) {$2$};
  \node[labelprio3] at (0,2.5) {$3$};
  \node[labelprio3] at (0.8,5.8) {$3$};
  
  \node[labelprio3] at (2.2,2.5) {$3$};
  \node[labelprio4] at (1.5,1.8) {$4$};
  \node<3->[labelprio3] at (1.55,1.8) {$3$};
  
%   \TState{3,9,15}{f_1, a_0, c_0, fc_2}
%   \TState{4,10}{f_1, a_1, c_0, fc_2}
%   \TState{5,11}{f_1, a_1, c_1, fc_2}
%   \TState{6,12}{f_1, a_1, c_1, fc_3}
%   \TState{7,13}{f_1, a_0, c_1, fc_3}
%   \TState{8,14}{f_1, a_0, c_0, fc_3}
\end{tikzpicture}

\pause[2]
\vspace*{-1cm}
%\vspace*{.3cm}
\bigskip
\begin{flushright}
  \f Memoryless (no accumulation)\\
  Unplayable action: $\PHfrappe{f_1}{f_1}{f_0}$
\end{flushright}

\end{frame}


\section{Static Analysis of Standard Process Hitting}
% Exemples de structures abstraites (graphes de causalité locale)

\begin{frame}
  \frametitle{Static analysis: successive reachability}
  \framesubtitle{\tcite{\cpmrmscs}}

Successive reachability of processes:

\begin{columns}
\begin{column}{0.55\textwidth}

\begin{center}
\scalebox{0.75}{
\begin{tikzpicture}
  %\path[use as bounding box] (-1,-3) rectangle (7,2);
  \exatt

  \TState{1-5}{a_0,b_0,c_0,d_0}
  \TState{6}{a_0,b_0,c_1,d_0}
  \TState{7}{a_0,b_0,c_1,d_1}
  \TState{8}{a_0,b_1,c_1,d_1}
  \node<9>[process,very thick] (d_2) at (d_2.center) {};
  \TState{9}{a_0,b_1,c_1,d_2}

  \node<2>[objective] at (d_1.center) {1?};
  \node<2>[objective] at (d_2.center) {2?};

  \node<3>[objective] at (d_1.center) {1?};
  \node<3>[objective] at (b_1.center) {2?};
  \node<3>[objective] at (d_2.center) {3?};

  \node<4-8>[objective] at (d_2.center) {1?};

  %\node<3>[process,very thick] (d_1) at (d_1.center) {1?};
  %\node<3>[process,very thick] (b_1) at (b_1.center) {2?};
  %\node<3>[process,very thick] (d_2) at (d_2.center) {3?};

  %\node<4-8>[process,very thick] (d_2) at (d_2.center) {1?};

  \only<5>{\THit{a_0}{hlhit}{c_0}{.north}{c_1}}
  \path<5>[bounce,bend left,hlhit] \TBounce{c_0}{}{c_1}{.west};
  \only<6>{\THit{b_0}{hlhit}{d_0}{.west}{d_1}}
  \path<6>[bounce,bend left,hlhit] \TBounce{d_0}{}{d_1}{.south};
  \only<7>{\THit{c_1}{bend left=20pt,hlhit}{b_0}{.west}{b_1}}
  \path<7>[bounce,bend left,hlhit] \TBounce{b_0}{}{b_1}{.south};
  \only<8>{\THit{b_1}{hlhit}{d_1}{.west}{d_2}}
  \path<8>[bounce,bend left,hlhit] \TBounce{d_1}{}{d_2}{.south};
\end{tikzpicture}
}
\end{center}

\end{column}
\begin{column}{0.45\textwidth}

%\pause
~\\~\\
\begin{itemize}
  \item Initial state
    \\ \rex{\PHetat{a_1, b_0, c_0, d_0}} \pause
  \item Objectives
    \\ \rex{$[\ \Rsh d_1\ \PHconcat\ \Rsh d_2\ ]$} \pause
    \\\smallskip \rex{$[\ \Rsh d_1 \PHconcat\ \Rsh b_1 \PHconcat\ \Rsh d_2\ ]$} \pause
    \\\smallskip \rex{$[\ \Rsh d_2\ ]$} \pause
\end{itemize}

\end{column}
\end{columns}

\medskip
\begin{center}
\f Concretization of the objective = scenario

\ex{
$ \only<5>{\underline{\PHfrappe{a_0}{c_0}{c_1}}} \only<-4,6->{\PHfrappe{a_0}{c_0}{c_1}} \PHconcat %
  \only<6>{\underline{\PHfrappe{b_0}{d_0}{d_1}}}\only<-5,7->{\PHfrappe{b_0}{d_0}{d_1}} \PHconcat %
  \only<7>{\underline{\PHfrappe{c_1}{b_0}{b_1}}}\only<-6,8->{\PHfrappe{c_1}{b_0}{b_1}} \PHconcat %
  \only<8>{\underline{\PHfrappe{b_1}{d_1}{d_2}}}\only<-7,9->{\PHfrappe{b_1}{d_1}{d_2}}
$}
\end{center}
\end{frame}


% 
% \begin{frame}
%   \frametitle{Over- and Under-approximations}
%   \framesubtitle{\tcite{\cpmrmscs}}
% 
% Static analysis by abstractions:
% \begin{fleches}
%   \item Directly checking an objective sequence $R$ is hard
%   \item Rather check the approximations $P$ and $Q$, where \tval{$P \Rightarrow R \Rightarrow Q$}:
% \end{fleches}
% 
% \begin{center}
% \scalebox{0.6}{
% \figsa
% }
% \end{center}
% 
% \only<-7>{~}
% \only<8->{
% Polynomial w.r.t.~the number of sorts and \\exponential w.r.t.~the number of processes in each sort
% \begin{fleches}
%   \item Efficient for big models with few levels of expression
% \end{fleches}
% }
% \end{frame}



\begin{frame}
  \frametitle{Under-approximation}

\def \tu {2}
\def \tub {3}
\def \tuf {4}

\begin{columns}
\begin{column}{0.48\textwidth}

\begin{center}
\scalebox{0.55}{
\begin{tikzpicture}
  \exatt
  \TState{-\tu}{a_1,b_1,c_1,d_0}
  \TState{\tub-}{a_0,b_1,c_0,d_0}
  \node[objective] (d_2) at (d_2.center) {?};
\end{tikzpicture}
}
\end{center}

\end{column}
\begin{column}{0.52\textwidth}

\vspace{1.5em}
\tval{Sufficient condition}:

\smallskip
\begin{itemize}
  \item no cycle
  \item \only<-\tu>{each objective has a solution} \only<\tub->{\sout{each objective has a solution}}
\end{itemize}
\begin{center}
  \only<\tu>{\Large\textcolor{darkgreen}{$R$ is \textbf{true}}} \only<\tuf>{\Large\textcolor{darkyellow}{\textbf{Inconclusive}}}
\end{center}

\end{column}
\end{columns}

\begin{center}%
%\vspace*{1cm}%
\scalebox{\scaleex}{%
\only<-\tu>{%
\scalebox{\scaleex}{%
\begin{tikzpicture}[aS]
  \path[use as bounding box] (.7,1) rectangle (5.8,2.5);

  \glclegend{}{$d_2$}{$\PHobj{d_0}{d_2}$}
\end{tikzpicture}
}
  \sauyes
}
\only<\tub->{
  \sauinconc
}}
\end{center}
\end{frame}



\begin{frame}
  \frametitle{Over-approximation}

\def \to {4}
\def \tob {5}
\def \tof {6}
\def \tokp {7}

\begin{columns}
\begin{column}{0.48\textwidth}

\begin{center}
\scalebox{0.55}{
\begin{tikzpicture}
  \exatt
  \TState{-\to}{a_1,b_0,c_0,d_1}
  \TState{\tob-}{a_1,b_1,c_1,d_0}
  \node[objective] (d_2) at (d_2.center) {?};
\end{tikzpicture}
}
\end{center}
\bigskip

\end{column}
\begin{column}{0.52\textwidth}

\tval{Necessary condition}:

\smallskip
\only<2->{
\only<3-\to>{\sout{There exists a traversal}}\only<2,\tob->{There exists a traversal}
with no cycle

\smallskip
\begin{itemize}
  \item \only<3-\to>{\sout{objective $\rightarrow$ follow one solution}}\only<1-2,\tob->{objective $\rightarrow$ follow one solution}
  \item solution $\rightarrow$ follow all processes
  \item process $\rightarrow$ follow all objectives
\end{itemize}
\begin{center}
  \only<\to>{\Large\textcolor{red}{$R$ is \textbf{false}}}\only<\tof->{\Large\textcolor{darkyellow}{\textbf{Inconclusive}}}
\end{center}
}

\end{column}
\end{columns}

\begin{center}
\scalebox{\scaleex}{
\only<1-\to>{
  \saono
}
\only<\tob->{
  \saoinconc
}}
\end{center}
\end{frame}


% 
% \begin{frame}[c]
%   \frametitle{Implementation \& Execution times}
% 
% \Pint\tval{: Existing free OCaml library}
% 
% \medskip
% \f Compiler + tools for Process Hitting models
% 
% \f Documentation \& examples: \lien{http://processhitting.wordpress.com/}
% 
% \pause
% \bigskip
% \medskip
% \tval{Computation time for various reachability analyses:}
% 
% \medskip
% \small
% \begin{tabular}{r||c|c|c|c||c|c|c|}
% \hline
% \tval{Model} & Sorts & Procs & Actions & States & Biocham$^1$ & libddd$^2$ & \Pint \\\hline
% \tval{\ex{egfr20}} & 35 & 196 & 670 & $2^{64}$ & [3s--$\infty$] & [1s--150s] & \tval{0.007s} \\\hline
% \tval{\ex{tcrsig40}} & 54 & 156 & 301 & $2^{73}$ & [1s--$\infty$] & [0.6s--$\infty$] & \tval{0.004s} \\\hline
% \tval{\ex{tcrsig94}} & 133 & 448 & 1124 & $2^{194}$ & $\infty$ & $\infty$ & \tval{0.030s} \\\hline
% \tval{\ex{egfr104}} & 193 & 748 &  2356 & $2^{320}$ &  $\infty$ & $\infty$ & \tval{0.050s}\\\hline
% \end{tabular}
% 
% \medskip
% \quad$^1$ Inria Paris-Rocquencourt/Contraintes\\
% \quad$^2$ LIP6/Move
% 
% \cmodels
% \end{frame}

% Recherche de points fixes par n-cliques

\begin{frame}[c]
  \frametitle{Static Analysis: Fixed Points}
  \framesubtitle{\tcite{\cpmrtcsb}}

\tval{Fixed point} = state where no action can be fired
\begin{fleches}
  \item avoid couples of processes bounded by an action
\only<1>{\\\smallskip~}
\only<2->{\item Hitless Graph \only<3->{\f \tval{n-cliques} = fixed points}}
\end{fleches}

\bigskip
\begin{columns}
\begin{column}{0.5\textwidth}

\begin{center}
\scalebox{0.7}{
\begin{tikzpicture}
\exdefb
\exdefbfrappes
\end{tikzpicture}
}
\end{center}

\end{column}
\begin{column}{0.5\textwidth}

\only<2->{
\begin{center}
\scalebox{0.7}{
\tikzstyle{current process}=[process,fill=red]
\begin{tikzpicture}[hitless graph]
\exdefb
\exdefbsf
\end{tikzpicture}
\tikzstyle{current process}=[process,fill=blue]
}
\end{center}
}

\end{column}
\end{columns}

\bigskip
\pause[6]
Exponential complexity w.r.t.~the number of sorts

\end{frame}


\setcounter{framenumber}{\value{finalframe}}

\end{document}
